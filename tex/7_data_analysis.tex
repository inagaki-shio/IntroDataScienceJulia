% Default to the notebook output style

    


% Inherit from the specified cell style.




    
\documentclass[a4paper,dvipdfmx,uplatex]{jsarticle}
\usepackage[dvipdfmx]{hyperref}
\usepackage{pxjahyper}

    
    
    \usepackage[T1]{fontenc}
    % Nicer default font than Computer Modern for most use cases
    \usepackage{palatino}

    % Basic figure setup, for now with no caption control since it's done
    % automatically by Pandoc (which extracts ![](path) syntax from Markdown).
    \usepackage{graphicx}
    % We will generate all images so they have a width \maxwidth. This means
    % that they will get their normal width if they fit onto the page, but
    % are scaled down if they would overflow the margins.
    \makeatletter
    \def\maxwidth{\ifdim\Gin@nat@width>\linewidth\linewidth
    \else\Gin@nat@width\fi}
    \makeatother
    \let\Oldincludegraphics\includegraphics
    % Set max figure width to be 80% of text width, for now hardcoded.
    \renewcommand{\includegraphics}[1]{\Oldincludegraphics[width=.8\maxwidth]{#1}}
    % Ensure that by default, figures have no caption (until we provide a
    % proper Figure object with a Caption API and a way to capture that
    % in the conversion process - todo).
    \usepackage{caption}
    \DeclareCaptionLabelFormat{nolabel}{}
    \captionsetup{labelformat=nolabel}

    \usepackage{adjustbox} % Used to constrain images to a maximum size 
    \usepackage{xcolor} % Allow colors to be defined
    \usepackage{enumerate} % Needed for markdown enumerations to work
    \usepackage{geometry} % Used to adjust the document margins
    \usepackage{amsmath} % Equations
    \usepackage{amssymb} % Equations
    \usepackage{textcomp} % defines textquotesingle
    % Hack from http://tex.stackexchange.com/a/47451/13684:
    \AtBeginDocument{%
        \def\PYZsq{\textquotesingle}% Upright quotes in Pygmentized code
    }
    \usepackage{upquote} % Upright quotes for verbatim code
    \usepackage{eurosym} % defines \euro
    \usepackage[mathletters]{ucs} % Extended unicode (utf-8) support
    \usepackage[utf8x]{inputenc} % Allow utf-8 characters in the tex document
    \usepackage{fancyvrb} % verbatim replacement that allows latex
    \usepackage{grffile} % extends the file name processing of package graphics 
                         % to support a larger range 
    % The hyperref package gives us a pdf with properly built
    % internal navigation ('pdf bookmarks' for the table of contents,
    % internal cross-reference links, web links for URLs, etc.)
    \usepackage{hyperref}
    \usepackage{longtable} % longtable support required by pandoc >1.10
    \usepackage{booktabs}  % table support for pandoc > 1.12.2
    \usepackage[normalem]{ulem} % ulem is needed to support strikethroughs (\sout)
                                % normalem makes italics be italics, not underlines
    

    
    
    % Colors for the hyperref package
    \definecolor{urlcolor}{rgb}{0,.145,.698}
    \definecolor{linkcolor}{rgb}{.71,0.21,0.01}
    \definecolor{citecolor}{rgb}{.12,.54,.11}

    % ANSI colors
    \definecolor{ansi-black}{HTML}{3E424D}
    \definecolor{ansi-black-intense}{HTML}{282C36}
    \definecolor{ansi-red}{HTML}{E75C58}
    \definecolor{ansi-red-intense}{HTML}{B22B31}
    \definecolor{ansi-green}{HTML}{00A250}
    \definecolor{ansi-green-intense}{HTML}{007427}
    \definecolor{ansi-yellow}{HTML}{DDB62B}
    \definecolor{ansi-yellow-intense}{HTML}{B27D12}
    \definecolor{ansi-blue}{HTML}{208FFB}
    \definecolor{ansi-blue-intense}{HTML}{0065CA}
    \definecolor{ansi-magenta}{HTML}{D160C4}
    \definecolor{ansi-magenta-intense}{HTML}{A03196}
    \definecolor{ansi-cyan}{HTML}{60C6C8}
    \definecolor{ansi-cyan-intense}{HTML}{258F8F}
    \definecolor{ansi-white}{HTML}{C5C1B4}
    \definecolor{ansi-white-intense}{HTML}{A1A6B2}

    % commands and environments needed by pandoc snippets
    % extracted from the output of `pandoc -s`
    \providecommand{\tightlist}{%
      \setlength{\itemsep}{0pt}\setlength{\parskip}{0pt}}
    \DefineVerbatimEnvironment{Highlighting}{Verbatim}{commandchars=\\\{\}}
    % Add ',fontsize=\small' for more characters per line
    \newenvironment{Shaded}{}{}
    \newcommand{\KeywordTok}[1]{\textcolor[rgb]{0.00,0.44,0.13}{\textbf{{#1}}}}
    \newcommand{\DataTypeTok}[1]{\textcolor[rgb]{0.56,0.13,0.00}{{#1}}}
    \newcommand{\DecValTok}[1]{\textcolor[rgb]{0.25,0.63,0.44}{{#1}}}
    \newcommand{\BaseNTok}[1]{\textcolor[rgb]{0.25,0.63,0.44}{{#1}}}
    \newcommand{\FloatTok}[1]{\textcolor[rgb]{0.25,0.63,0.44}{{#1}}}
    \newcommand{\CharTok}[1]{\textcolor[rgb]{0.25,0.44,0.63}{{#1}}}
    \newcommand{\StringTok}[1]{\textcolor[rgb]{0.25,0.44,0.63}{{#1}}}
    \newcommand{\CommentTok}[1]{\textcolor[rgb]{0.38,0.63,0.69}{\textit{{#1}}}}
    \newcommand{\OtherTok}[1]{\textcolor[rgb]{0.00,0.44,0.13}{{#1}}}
    \newcommand{\AlertTok}[1]{\textcolor[rgb]{1.00,0.00,0.00}{\textbf{{#1}}}}
    \newcommand{\FunctionTok}[1]{\textcolor[rgb]{0.02,0.16,0.49}{{#1}}}
    \newcommand{\RegionMarkerTok}[1]{{#1}}
    \newcommand{\ErrorTok}[1]{\textcolor[rgb]{1.00,0.00,0.00}{\textbf{{#1}}}}
    \newcommand{\NormalTok}[1]{{#1}}
    
    % Additional commands for more recent versions of Pandoc
    \newcommand{\ConstantTok}[1]{\textcolor[rgb]{0.53,0.00,0.00}{{#1}}}
    \newcommand{\SpecialCharTok}[1]{\textcolor[rgb]{0.25,0.44,0.63}{{#1}}}
    \newcommand{\VerbatimStringTok}[1]{\textcolor[rgb]{0.25,0.44,0.63}{{#1}}}
    \newcommand{\SpecialStringTok}[1]{\textcolor[rgb]{0.73,0.40,0.53}{{#1}}}
    \newcommand{\ImportTok}[1]{{#1}}
    \newcommand{\DocumentationTok}[1]{\textcolor[rgb]{0.73,0.13,0.13}{\textit{{#1}}}}
    \newcommand{\AnnotationTok}[1]{\textcolor[rgb]{0.38,0.63,0.69}{\textbf{\textit{{#1}}}}}
    \newcommand{\CommentVarTok}[1]{\textcolor[rgb]{0.38,0.63,0.69}{\textbf{\textit{{#1}}}}}
    \newcommand{\VariableTok}[1]{\textcolor[rgb]{0.10,0.09,0.49}{{#1}}}
    \newcommand{\ControlFlowTok}[1]{\textcolor[rgb]{0.00,0.44,0.13}{\textbf{{#1}}}}
    \newcommand{\OperatorTok}[1]{\textcolor[rgb]{0.40,0.40,0.40}{{#1}}}
    \newcommand{\BuiltInTok}[1]{{#1}}
    \newcommand{\ExtensionTok}[1]{{#1}}
    \newcommand{\PreprocessorTok}[1]{\textcolor[rgb]{0.74,0.48,0.00}{{#1}}}
    \newcommand{\AttributeTok}[1]{\textcolor[rgb]{0.49,0.56,0.16}{{#1}}}
    \newcommand{\InformationTok}[1]{\textcolor[rgb]{0.38,0.63,0.69}{\textbf{\textit{{#1}}}}}
    \newcommand{\WarningTok}[1]{\textcolor[rgb]{0.38,0.63,0.69}{\textbf{\textit{{#1}}}}}
    
    
    % Define a nice break command that doesn't care if a line doesn't already
    % exist.
    \def\br{\hspace*{\fill} \\* }
    % Math Jax compatability definitions
    \def\gt{>}
    \def\lt{<}
    % Document parameters
    \title{7\_data\_analysis}
    
    
    

    % Pygments definitions
    
\makeatletter
\def\PY@reset{\let\PY@it=\relax \let\PY@bf=\relax%
    \let\PY@ul=\relax \let\PY@tc=\relax%
    \let\PY@bc=\relax \let\PY@ff=\relax}
\def\PY@tok#1{\csname PY@tok@#1\endcsname}
\def\PY@toks#1+{\ifx\relax#1\empty\else%
    \PY@tok{#1}\expandafter\PY@toks\fi}
\def\PY@do#1{\PY@bc{\PY@tc{\PY@ul{%
    \PY@it{\PY@bf{\PY@ff{#1}}}}}}}
\def\PY#1#2{\PY@reset\PY@toks#1+\relax+\PY@do{#2}}

\expandafter\def\csname PY@tok@cm\endcsname{\let\PY@it=\textit\def\PY@tc##1{\textcolor[rgb]{0.25,0.50,0.50}{##1}}}
\expandafter\def\csname PY@tok@sx\endcsname{\def\PY@tc##1{\textcolor[rgb]{0.00,0.50,0.00}{##1}}}
\expandafter\def\csname PY@tok@vc\endcsname{\def\PY@tc##1{\textcolor[rgb]{0.10,0.09,0.49}{##1}}}
\expandafter\def\csname PY@tok@vg\endcsname{\def\PY@tc##1{\textcolor[rgb]{0.10,0.09,0.49}{##1}}}
\expandafter\def\csname PY@tok@ni\endcsname{\let\PY@bf=\textbf\def\PY@tc##1{\textcolor[rgb]{0.60,0.60,0.60}{##1}}}
\expandafter\def\csname PY@tok@cpf\endcsname{\let\PY@it=\textit\def\PY@tc##1{\textcolor[rgb]{0.25,0.50,0.50}{##1}}}
\expandafter\def\csname PY@tok@sc\endcsname{\def\PY@tc##1{\textcolor[rgb]{0.73,0.13,0.13}{##1}}}
\expandafter\def\csname PY@tok@gt\endcsname{\def\PY@tc##1{\textcolor[rgb]{0.00,0.27,0.87}{##1}}}
\expandafter\def\csname PY@tok@no\endcsname{\def\PY@tc##1{\textcolor[rgb]{0.53,0.00,0.00}{##1}}}
\expandafter\def\csname PY@tok@k\endcsname{\let\PY@bf=\textbf\def\PY@tc##1{\textcolor[rgb]{0.00,0.50,0.00}{##1}}}
\expandafter\def\csname PY@tok@w\endcsname{\def\PY@tc##1{\textcolor[rgb]{0.73,0.73,0.73}{##1}}}
\expandafter\def\csname PY@tok@mh\endcsname{\def\PY@tc##1{\textcolor[rgb]{0.40,0.40,0.40}{##1}}}
\expandafter\def\csname PY@tok@il\endcsname{\def\PY@tc##1{\textcolor[rgb]{0.40,0.40,0.40}{##1}}}
\expandafter\def\csname PY@tok@ne\endcsname{\let\PY@bf=\textbf\def\PY@tc##1{\textcolor[rgb]{0.82,0.25,0.23}{##1}}}
\expandafter\def\csname PY@tok@mb\endcsname{\def\PY@tc##1{\textcolor[rgb]{0.40,0.40,0.40}{##1}}}
\expandafter\def\csname PY@tok@s2\endcsname{\def\PY@tc##1{\textcolor[rgb]{0.73,0.13,0.13}{##1}}}
\expandafter\def\csname PY@tok@ow\endcsname{\let\PY@bf=\textbf\def\PY@tc##1{\textcolor[rgb]{0.67,0.13,1.00}{##1}}}
\expandafter\def\csname PY@tok@gs\endcsname{\let\PY@bf=\textbf}
\expandafter\def\csname PY@tok@kn\endcsname{\let\PY@bf=\textbf\def\PY@tc##1{\textcolor[rgb]{0.00,0.50,0.00}{##1}}}
\expandafter\def\csname PY@tok@ge\endcsname{\let\PY@it=\textit}
\expandafter\def\csname PY@tok@sh\endcsname{\def\PY@tc##1{\textcolor[rgb]{0.73,0.13,0.13}{##1}}}
\expandafter\def\csname PY@tok@go\endcsname{\def\PY@tc##1{\textcolor[rgb]{0.53,0.53,0.53}{##1}}}
\expandafter\def\csname PY@tok@c\endcsname{\let\PY@it=\textit\def\PY@tc##1{\textcolor[rgb]{0.25,0.50,0.50}{##1}}}
\expandafter\def\csname PY@tok@c1\endcsname{\let\PY@it=\textit\def\PY@tc##1{\textcolor[rgb]{0.25,0.50,0.50}{##1}}}
\expandafter\def\csname PY@tok@gp\endcsname{\let\PY@bf=\textbf\def\PY@tc##1{\textcolor[rgb]{0.00,0.00,0.50}{##1}}}
\expandafter\def\csname PY@tok@nt\endcsname{\let\PY@bf=\textbf\def\PY@tc##1{\textcolor[rgb]{0.00,0.50,0.00}{##1}}}
\expandafter\def\csname PY@tok@err\endcsname{\def\PY@bc##1{\setlength{\fboxsep}{0pt}\fcolorbox[rgb]{1.00,0.00,0.00}{1,1,1}{\strut ##1}}}
\expandafter\def\csname PY@tok@mo\endcsname{\def\PY@tc##1{\textcolor[rgb]{0.40,0.40,0.40}{##1}}}
\expandafter\def\csname PY@tok@nn\endcsname{\let\PY@bf=\textbf\def\PY@tc##1{\textcolor[rgb]{0.00,0.00,1.00}{##1}}}
\expandafter\def\csname PY@tok@nl\endcsname{\def\PY@tc##1{\textcolor[rgb]{0.63,0.63,0.00}{##1}}}
\expandafter\def\csname PY@tok@m\endcsname{\def\PY@tc##1{\textcolor[rgb]{0.40,0.40,0.40}{##1}}}
\expandafter\def\csname PY@tok@gi\endcsname{\def\PY@tc##1{\textcolor[rgb]{0.00,0.63,0.00}{##1}}}
\expandafter\def\csname PY@tok@mi\endcsname{\def\PY@tc##1{\textcolor[rgb]{0.40,0.40,0.40}{##1}}}
\expandafter\def\csname PY@tok@s1\endcsname{\def\PY@tc##1{\textcolor[rgb]{0.73,0.13,0.13}{##1}}}
\expandafter\def\csname PY@tok@vi\endcsname{\def\PY@tc##1{\textcolor[rgb]{0.10,0.09,0.49}{##1}}}
\expandafter\def\csname PY@tok@cs\endcsname{\let\PY@it=\textit\def\PY@tc##1{\textcolor[rgb]{0.25,0.50,0.50}{##1}}}
\expandafter\def\csname PY@tok@se\endcsname{\let\PY@bf=\textbf\def\PY@tc##1{\textcolor[rgb]{0.73,0.40,0.13}{##1}}}
\expandafter\def\csname PY@tok@kt\endcsname{\def\PY@tc##1{\textcolor[rgb]{0.69,0.00,0.25}{##1}}}
\expandafter\def\csname PY@tok@kd\endcsname{\let\PY@bf=\textbf\def\PY@tc##1{\textcolor[rgb]{0.00,0.50,0.00}{##1}}}
\expandafter\def\csname PY@tok@cp\endcsname{\def\PY@tc##1{\textcolor[rgb]{0.74,0.48,0.00}{##1}}}
\expandafter\def\csname PY@tok@nc\endcsname{\let\PY@bf=\textbf\def\PY@tc##1{\textcolor[rgb]{0.00,0.00,1.00}{##1}}}
\expandafter\def\csname PY@tok@na\endcsname{\def\PY@tc##1{\textcolor[rgb]{0.49,0.56,0.16}{##1}}}
\expandafter\def\csname PY@tok@gr\endcsname{\def\PY@tc##1{\textcolor[rgb]{1.00,0.00,0.00}{##1}}}
\expandafter\def\csname PY@tok@ss\endcsname{\def\PY@tc##1{\textcolor[rgb]{0.10,0.09,0.49}{##1}}}
\expandafter\def\csname PY@tok@sr\endcsname{\def\PY@tc##1{\textcolor[rgb]{0.73,0.40,0.53}{##1}}}
\expandafter\def\csname PY@tok@sd\endcsname{\let\PY@it=\textit\def\PY@tc##1{\textcolor[rgb]{0.73,0.13,0.13}{##1}}}
\expandafter\def\csname PY@tok@kc\endcsname{\let\PY@bf=\textbf\def\PY@tc##1{\textcolor[rgb]{0.00,0.50,0.00}{##1}}}
\expandafter\def\csname PY@tok@gu\endcsname{\let\PY@bf=\textbf\def\PY@tc##1{\textcolor[rgb]{0.50,0.00,0.50}{##1}}}
\expandafter\def\csname PY@tok@s\endcsname{\def\PY@tc##1{\textcolor[rgb]{0.73,0.13,0.13}{##1}}}
\expandafter\def\csname PY@tok@kp\endcsname{\def\PY@tc##1{\textcolor[rgb]{0.00,0.50,0.00}{##1}}}
\expandafter\def\csname PY@tok@mf\endcsname{\def\PY@tc##1{\textcolor[rgb]{0.40,0.40,0.40}{##1}}}
\expandafter\def\csname PY@tok@gh\endcsname{\let\PY@bf=\textbf\def\PY@tc##1{\textcolor[rgb]{0.00,0.00,0.50}{##1}}}
\expandafter\def\csname PY@tok@nv\endcsname{\def\PY@tc##1{\textcolor[rgb]{0.10,0.09,0.49}{##1}}}
\expandafter\def\csname PY@tok@nb\endcsname{\def\PY@tc##1{\textcolor[rgb]{0.00,0.50,0.00}{##1}}}
\expandafter\def\csname PY@tok@kr\endcsname{\let\PY@bf=\textbf\def\PY@tc##1{\textcolor[rgb]{0.00,0.50,0.00}{##1}}}
\expandafter\def\csname PY@tok@gd\endcsname{\def\PY@tc##1{\textcolor[rgb]{0.63,0.00,0.00}{##1}}}
\expandafter\def\csname PY@tok@nf\endcsname{\def\PY@tc##1{\textcolor[rgb]{0.00,0.00,1.00}{##1}}}
\expandafter\def\csname PY@tok@nd\endcsname{\def\PY@tc##1{\textcolor[rgb]{0.67,0.13,1.00}{##1}}}
\expandafter\def\csname PY@tok@sb\endcsname{\def\PY@tc##1{\textcolor[rgb]{0.73,0.13,0.13}{##1}}}
\expandafter\def\csname PY@tok@o\endcsname{\def\PY@tc##1{\textcolor[rgb]{0.40,0.40,0.40}{##1}}}
\expandafter\def\csname PY@tok@ch\endcsname{\let\PY@it=\textit\def\PY@tc##1{\textcolor[rgb]{0.25,0.50,0.50}{##1}}}
\expandafter\def\csname PY@tok@si\endcsname{\let\PY@bf=\textbf\def\PY@tc##1{\textcolor[rgb]{0.73,0.40,0.53}{##1}}}
\expandafter\def\csname PY@tok@bp\endcsname{\def\PY@tc##1{\textcolor[rgb]{0.00,0.50,0.00}{##1}}}

\def\PYZbs{\char`\\}
\def\PYZus{\char`\_}
\def\PYZob{\char`\{}
\def\PYZcb{\char`\}}
\def\PYZca{\char`\^}
\def\PYZam{\char`\&}
\def\PYZlt{\char`\<}
\def\PYZgt{\char`\>}
\def\PYZsh{\char`\#}
\def\PYZpc{\char`\%}
\def\PYZdl{\char`\$}
\def\PYZhy{\char`\-}
\def\PYZsq{\char`\'}
\def\PYZdq{\char`\"}
\def\PYZti{\char`\~}
% for compatibility with earlier versions
\def\PYZat{@}
\def\PYZlb{[}
\def\PYZrb{]}
\makeatother


    % Exact colors from NB
    \definecolor{incolor}{rgb}{0.0, 0.0, 0.5}
    \definecolor{outcolor}{rgb}{0.545, 0.0, 0.0}



    
    % Prevent overflowing lines due to hard-to-break entities
    \sloppy 
    % Setup hyperref package
    \hypersetup{
      breaklinks=true,  % so long urls are correctly broken across lines
      colorlinks=true,
      urlcolor=urlcolor,
      linkcolor=linkcolor,
      citecolor=citecolor,
      }
    % Slightly bigger margins than the latex defaults
    
    \geometry{verbose,tmargin=1in,bmargin=1in,lmargin=1in,rmargin=1in}
    
    

    \begin{document}
    
    
    \maketitle
    
    

    
    \section{Introduction to Data Science with
Julia}\label{introduction-to-data-science-with-julia}

\section{目次}\label{ux76eeux6b21}

\begin{itemize}
\tightlist
\item
  \protect\hyperlink{ux7d9aux30c7ux30fcux30bfux5206ux6790}{続データ分析}
\item
  \protect\hyperlink{Kaggle}{Kaggle}
\item
  \protect\hyperlink{ux7df4ux7fd2ux554fux984c}{練習問題}
\end{itemize}

    \section{続データ分析}\label{ux7d9aux30c7ux30fcux30bfux5206ux6790}

ここでは DataFrames を使ってデータ分析するときに便利な関数を紹介します。

    \begin{Verbatim}[commandchars=\\\{\}]
{\color{incolor}In [{\color{incolor}1}]:} \PY{k}{import} \PY{n}{RDatasets}\PY{p}{,} \PY{n}{DataFrames}
        \PY{n}{anscombe} \PY{o}{=} \PY{n}{RDatasets}\PY{o}{.}\PY{n}{dataset}\PY{p}{(}\PY{l+s}{\PYZdq{}}\PY{l+s}{datasets}\PY{l+s}{\PYZdq{}}\PY{p}{,}\PY{l+s}{\PYZdq{}}\PY{l+s}{anscombe}\PY{l+s}{\PYZdq{}}\PY{p}{)}
\end{Verbatim}

            \begin{Verbatim}[commandchars=\\\{\}]
{\color{outcolor}Out[{\color{outcolor}1}]:} 11×8 DataFrames.DataFrame
        │ Row │ X1 │ X2 │ X3 │ X4 │ Y1    │ Y2   │ Y3    │ Y4   │
        ├─────┼────┼────┼────┼────┼───────┼──────┼───────┼──────┤
        │ 1   │ 10 │ 10 │ 10 │ 8  │ 8.04  │ 9.14 │ 7.46  │ 6.58 │
        │ 2   │ 8  │ 8  │ 8  │ 8  │ 6.95  │ 8.14 │ 6.77  │ 5.76 │
        │ 3   │ 13 │ 13 │ 13 │ 8  │ 7.58  │ 8.74 │ 12.74 │ 7.71 │
        │ 4   │ 9  │ 9  │ 9  │ 8  │ 8.81  │ 8.77 │ 7.11  │ 8.84 │
        │ 5   │ 11 │ 11 │ 11 │ 8  │ 8.33  │ 9.26 │ 7.81  │ 8.47 │
        │ 6   │ 14 │ 14 │ 14 │ 8  │ 9.96  │ 8.1  │ 8.84  │ 7.04 │
        │ 7   │ 6  │ 6  │ 6  │ 8  │ 7.24  │ 6.13 │ 6.08  │ 5.25 │
        │ 8   │ 4  │ 4  │ 4  │ 19 │ 4.26  │ 3.1  │ 5.39  │ 12.5 │
        │ 9   │ 12 │ 12 │ 12 │ 8  │ 10.84 │ 9.13 │ 8.15  │ 5.56 │
        │ 10  │ 7  │ 7  │ 7  │ 8  │ 4.82  │ 7.26 │ 6.42  │ 7.91 │
        │ 11  │ 5  │ 5  │ 5  │ 8  │ 5.68  │ 4.74 │ 5.73  │ 6.89 │
\end{Verbatim}
        
    各列の平均などは describe を使うと便利でした。

    \begin{Verbatim}[commandchars=\\\{\}]
{\color{incolor}In [{\color{incolor}2}]:} \PY{n}{DataFrames}\PY{o}{.}\PY{n}{describe}\PY{p}{(}\PY{n}{anscombe}\PY{p}{)}
\end{Verbatim}

    \begin{Verbatim}[commandchars=\\\{\}]
X1
Min      4.0
1st Qu.  6.5
Median   9.0
Mean     9.0
3rd Qu.  11.5
Max      14.0
NAs      0
NA\%      0.0\%

X2
Min      4.0
1st Qu.  6.5
Median   9.0
Mean     9.0
3rd Qu.  11.5
Max      14.0
NAs      0
NA\%      0.0\%

X3
Min      4.0
1st Qu.  6.5
Median   9.0
Mean     9.0
3rd Qu.  11.5
Max      14.0
NAs      0
NA\%      0.0\%

X4
Min      8.0
1st Qu.  8.0
Median   8.0
Mean     9.0
3rd Qu.  8.0
Max      19.0
NAs      0
NA\%      0.0\%

Y1
Min      4.26
1st Qu.  6.3149999999999995
Median   7.58
Mean     7.500909090909093
3rd Qu.  8.57
Max      10.84
NAs      0
NA\%      0.0\%

Y2
Min      3.1
1st Qu.  6.695
Median   8.14
Mean     7.500909090909091
3rd Qu.  8.95
Max      9.26
NAs      0
NA\%      0.0\%

Y3
Min      5.39
1st Qu.  6.25
Median   7.11
Mean     7.500000000000001
3rd Qu.  7.98
Max      12.74
NAs      0
NA\%      0.0\%

Y4
Min      5.25
1st Qu.  6.17
Median   7.04
Mean     7.50090909090909
3rd Qu.  8.190000000000001
Max      12.5
NAs      0
NA\%      0.0\%


    \end{Verbatim}

    各列ごとに関数を作用させる場合は colwise を使います

基本文法

\begin{Shaded}
\begin{Highlighting}[]
    \NormalTok{DataFrames.colwise(}\KeywordTok{function}\NormalTok{, DataFrame)}
\end{Highlighting}
\end{Shaded}

    \begin{Verbatim}[commandchars=\\\{\}]
{\color{incolor}In [{\color{incolor}3}]:} \PY{n}{DataFrames}\PY{o}{.}\PY{n}{colwise}\PY{p}{(}\PY{n}{mean}\PY{p}{,} \PY{n}{anscombe}\PY{p}{)} \PY{c}{\PYZsh{} 各列の平均}
\end{Verbatim}

            \begin{Verbatim}[commandchars=\\\{\}]
{\color{outcolor}Out[{\color{outcolor}3}]:} 8-element Array\{Any,1\}:
         [9.0]    
         [9.0]    
         [9.0]    
         [9.0]    
         [7.50091]
         [7.50091]
         [7.5]    
         [7.50091]
\end{Verbatim}
        
    特定の列に対してだけ関数を作用させたい場合は、anscombe{[}{[}:X1,
:Y1{]}{]} のように作用させる列を明示的に指定します。

    \begin{Verbatim}[commandchars=\\\{\}]
{\color{incolor}In [{\color{incolor}4}]:} \PY{n}{DataFrames}\PY{o}{.}\PY{n}{colwise}\PY{p}{(}\PY{n}{mean}\PY{p}{,} \PY{n}{anscombe}\PY{p}{[}\PY{p}{[}\PY{p}{:}\PY{n}{X1}\PY{p}{,} \PY{p}{:}\PY{n}{Y1}\PY{p}{]}\PY{p}{]}\PY{p}{)}
\end{Verbatim}

            \begin{Verbatim}[commandchars=\\\{\}]
{\color{outcolor}Out[{\color{outcolor}4}]:} 2-element Array\{Any,1\}:
         [9.0]    
         [7.50091]
\end{Verbatim}
        
    \begin{Verbatim}[commandchars=\\\{\}]
{\color{incolor}In [{\color{incolor} }]:} 
\end{Verbatim}

    eachrow, eachcol は、for 文を使って各行・列を抜き出すときに便利です。

同様のことは行番号や列番号を指定しても出来ますが、よくあるバグとして
df{[}i,:{]} と書くべきところを df{[}:,i{]}
と書いてしまうといったこともあるので eachrow, eachcol
を使ったほうが良いと思います。

    \begin{Verbatim}[commandchars=\\\{\}]
{\color{incolor}In [{\color{incolor}5}]:} \PY{k}{for} \PY{n}{row} \PY{k}{in} \PY{n}{DataFrames}\PY{o}{.}\PY{n}{eachrow}\PY{p}{(}\PY{n}{anscombe}\PY{p}{)}
            \PY{n}{println}\PY{p}{(}\PY{n}{row}\PY{p}{)}
        \PY{k}{end}
\end{Verbatim}

    \begin{Verbatim}[commandchars=\\\{\}]
DataFrameRow (row 1)
X1  10
X2  10
X3  10
X4  8
Y1  8.04
Y2  9.14
Y3  7.46
Y4  6.58

DataFrameRow (row 2)
X1  8
X2  8
X3  8
X4  8
Y1  6.95
Y2  8.14
Y3  6.77
Y4  5.76

DataFrameRow (row 3)
X1  13
X2  13
X3  13
X4  8
Y1  7.58
Y2  8.74
Y3  12.74
Y4  7.71

DataFrameRow (row 4)
X1  9
X2  9
X3  9
X4  8
Y1  8.81
Y2  8.77
Y3  7.11
Y4  8.84

DataFrameRow (row 5)
X1  11
X2  11
X3  11
X4  8
Y1  8.33
Y2  9.26
Y3  7.81
Y4  8.47

DataFrameRow (row 6)
X1  14
X2  14
X3  14
X4  8
Y1  9.96
Y2  8.1
Y3  8.84
Y4  7.04

DataFrameRow (row 7)
X1  6
X2  6
X3  6
X4  8
Y1  7.24
Y2  6.13
Y3  6.08
Y4  5.25

DataFrameRow (row 8)
X1  4
X2  4
X3  4
X4  19
Y1  4.26
Y2  3.1
Y3  5.39
Y4  12.5

DataFrameRow (row 9)
X1  12
X2  12
X3  12
X4  8
Y1  10.84
Y2  9.13
Y3  8.15
Y4  5.56

DataFrameRow (row 10)
X1  7
X2  7
X3  7
X4  8
Y1  4.82
Y2  7.26
Y3  6.42
Y4  7.91

DataFrameRow (row 11)
X1  5
X2  5
X3  5
X4  8
Y1  5.68
Y2  4.74
Y3  5.73
Y4  6.89


    \end{Verbatim}

    \begin{Verbatim}[commandchars=\\\{\}]
{\color{incolor}In [{\color{incolor}6}]:} \PY{k}{for} \PY{n}{col} \PY{k}{in} \PY{n}{DataFrames}\PY{o}{.}\PY{n}{eachcol}\PY{p}{(}\PY{n}{anscombe}\PY{p}{)}
            \PY{n}{println}\PY{p}{(}\PY{n}{col}\PY{p}{)}
        \PY{k}{end}
\end{Verbatim}

    \begin{Verbatim}[commandchars=\\\{\}]
(:X1,[10,8,13,9,11,14,6,4,12,7,5])
(:X2,[10,8,13,9,11,14,6,4,12,7,5])
(:X3,[10,8,13,9,11,14,6,4,12,7,5])
(:X4,[8,8,8,8,8,8,8,19,8,8,8])
(:Y1,[8.04,6.95,7.58,8.81,8.33,9.96,7.24,4.26,10.84,4.82,5.68])
(:Y2,[9.14,8.14,8.74,8.77,9.26,8.1,6.13,3.1,9.13,7.26,4.74])
(:Y3,[7.46,6.77,12.74,7.11,7.81,8.84,6.08,5.39,8.15,6.42,5.73])
(:Y4,[6.58,5.76,7.71,8.84,8.47,7.04,5.25,12.5,5.56,7.91,6.89])

    \end{Verbatim}

    \begin{Verbatim}[commandchars=\\\{\}]
{\color{incolor}In [{\color{incolor} }]:} 
\end{Verbatim}

    eachrow を使って試験 1, 2 両方の点数が 60 点以上の学生の ID
を抜き出すには次のようになります。

    \begin{Verbatim}[commandchars=\\\{\}]
{\color{incolor}In [{\color{incolor}7}]:} \PY{n}{scores} \PY{o}{=} \PY{n}{DataFrames}\PY{o}{.}\PY{n}{readtable}\PY{p}{(}\PY{l+s}{\PYZdq{}}\PY{l+s}{../data/scores.csv}\PY{l+s}{\PYZdq{}}\PY{p}{)}
        \PY{n}{DataFrames}\PY{o}{.}\PY{n}{head}\PY{p}{(}\PY{n}{scores}\PY{p}{)}
        \PY{n}{ID}  \PY{o}{=} \PY{k+kt}{Int}\PY{p}{[}\PY{p}{]}
        \PY{k}{for} \PY{n}{student} \PY{k}{in} \PY{n}{DataFrames}\PY{o}{.}\PY{n}{eachrow}\PY{p}{(}\PY{n}{scores}\PY{p}{)}
            \PY{k}{if} \PY{n}{student}\PY{p}{[}\PY{p}{:}\PY{n}{exam1}\PY{p}{]} \PY{o}{\PYZgt{}=} \PY{l+m+mi}{60} \PY{o}{\PYZam{}\PYZam{}} \PY{n}{student}\PY{p}{[}\PY{p}{:}\PY{n}{exam2}\PY{p}{]} \PY{o}{\PYZgt{}=} \PY{l+m+mi}{60}
                \PY{n}{push!}\PY{p}{(}\PY{n}{ID}\PY{p}{,} \PY{n}{student}\PY{p}{[}\PY{p}{:}\PY{n}{ID}\PY{p}{]}\PY{p}{)}
            \PY{k}{end}
        \PY{k}{end}
        \PY{p}{@}\PY{n}{show} \PY{n}{ID}\PY{p}{;}
\end{Verbatim}

    \begin{Verbatim}[commandchars=\\\{\}]
ID = [4,7,9,12,16,23,26,28,30,32,42,43,45,48,56,57,61,68,69,75,78,79,81,82,83,85,86,88,89,91,95,97,99,100]

    \end{Verbatim}

    \begin{Verbatim}[commandchars=\\\{\}]
{\color{incolor}In [{\color{incolor} }]:} 
\end{Verbatim}

    \protect\hyperlink{ux76eeux6b21}{目次に戻る}

    \section{Kaggle}\label{kaggle}

以上まででデータ分析に必要な基礎は身についたでしょう。これからはより実践的なデータに触れていきましょう。

とはいえ、統計局が出しているデータなどを用いて分析せよと言われても何をどうしたら良いのかわからないと思うので、\href{https://www.kaggle.com/}{Kaggle}
のチュートリアルをやってみましょう。 Kaggle
とはデータサイエンティスト達が集うコンペティションサイトです。問題によっては懸賞金が掛かっており、中には総額数億円掛かっている問題もあります。

今回はそんな Kaggel の中にチュートリアルとしてある
\href{https://www.kaggle.com/c/titanic}{Titanic: Machine Learning from
Disaster} をやってみましょう。Kaggle
を利用するには利用登録する必要が有ります。ですが JuliaBox 同様 Sign Up
で Google のアカウントを選べばすぐに終わります。

副題に Machine Learning
と入っていますが、機械学習を使わなくても参加できるので心配しないでください。

    \protect\hyperlink{ux76eeux6b21}{目次に戻る}


    % Add a bibliography block to the postdoc
    
    
    
    \end{document}
