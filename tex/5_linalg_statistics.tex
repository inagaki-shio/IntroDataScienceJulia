% Default to the notebook output style

    


% Inherit from the specified cell style.




    
\documentclass[a4paper,dvipdfmx,uplatex]{jsarticle}
\usepackage[dvipdfmx]{hyperref}
\usepackage{pxjahyper}

    
    
    \usepackage[T1]{fontenc}
    % Nicer default font than Computer Modern for most use cases
    \usepackage{palatino}

    % Basic figure setup, for now with no caption control since it's done
    % automatically by Pandoc (which extracts ![](path) syntax from Markdown).
    \usepackage{graphicx}
    % We will generate all images so they have a width \maxwidth. This means
    % that they will get their normal width if they fit onto the page, but
    % are scaled down if they would overflow the margins.
    \makeatletter
    \def\maxwidth{\ifdim\Gin@nat@width>\linewidth\linewidth
    \else\Gin@nat@width\fi}
    \makeatother
    \let\Oldincludegraphics\includegraphics
    % Set max figure width to be 80% of text width, for now hardcoded.
    \renewcommand{\includegraphics}[1]{\Oldincludegraphics[width=.8\maxwidth]{#1}}
    % Ensure that by default, figures have no caption (until we provide a
    % proper Figure object with a Caption API and a way to capture that
    % in the conversion process - todo).
    \usepackage{caption}
    \DeclareCaptionLabelFormat{nolabel}{}
    \captionsetup{labelformat=nolabel}

    \usepackage{adjustbox} % Used to constrain images to a maximum size 
    \usepackage{xcolor} % Allow colors to be defined
    \usepackage{enumerate} % Needed for markdown enumerations to work
    \usepackage{geometry} % Used to adjust the document margins
    \usepackage{amsmath} % Equations
    \usepackage{amssymb} % Equations
    \usepackage{textcomp} % defines textquotesingle
    % Hack from http://tex.stackexchange.com/a/47451/13684:
    \AtBeginDocument{%
        \def\PYZsq{\textquotesingle}% Upright quotes in Pygmentized code
    }
    \usepackage{upquote} % Upright quotes for verbatim code
    \usepackage{eurosym} % defines \euro
    \usepackage[mathletters]{ucs} % Extended unicode (utf-8) support
    \usepackage[utf8x]{inputenc} % Allow utf-8 characters in the tex document
    \usepackage{fancyvrb} % verbatim replacement that allows latex
    \usepackage{grffile} % extends the file name processing of package graphics 
                         % to support a larger range 
    % The hyperref package gives us a pdf with properly built
    % internal navigation ('pdf bookmarks' for the table of contents,
    % internal cross-reference links, web links for URLs, etc.)
    \usepackage{hyperref}
    \usepackage{longtable} % longtable support required by pandoc >1.10
    \usepackage{booktabs}  % table support for pandoc > 1.12.2
    \usepackage[normalem]{ulem} % ulem is needed to support strikethroughs (\sout)
                                % normalem makes italics be italics, not underlines
    

    
    
    % Colors for the hyperref package
    \definecolor{urlcolor}{rgb}{0,.145,.698}
    \definecolor{linkcolor}{rgb}{.71,0.21,0.01}
    \definecolor{citecolor}{rgb}{.12,.54,.11}

    % ANSI colors
    \definecolor{ansi-black}{HTML}{3E424D}
    \definecolor{ansi-black-intense}{HTML}{282C36}
    \definecolor{ansi-red}{HTML}{E75C58}
    \definecolor{ansi-red-intense}{HTML}{B22B31}
    \definecolor{ansi-green}{HTML}{00A250}
    \definecolor{ansi-green-intense}{HTML}{007427}
    \definecolor{ansi-yellow}{HTML}{DDB62B}
    \definecolor{ansi-yellow-intense}{HTML}{B27D12}
    \definecolor{ansi-blue}{HTML}{208FFB}
    \definecolor{ansi-blue-intense}{HTML}{0065CA}
    \definecolor{ansi-magenta}{HTML}{D160C4}
    \definecolor{ansi-magenta-intense}{HTML}{A03196}
    \definecolor{ansi-cyan}{HTML}{60C6C8}
    \definecolor{ansi-cyan-intense}{HTML}{258F8F}
    \definecolor{ansi-white}{HTML}{C5C1B4}
    \definecolor{ansi-white-intense}{HTML}{A1A6B2}

    % commands and environments needed by pandoc snippets
    % extracted from the output of `pandoc -s`
    \providecommand{\tightlist}{%
      \setlength{\itemsep}{0pt}\setlength{\parskip}{0pt}}
    \DefineVerbatimEnvironment{Highlighting}{Verbatim}{commandchars=\\\{\}}
    % Add ',fontsize=\small' for more characters per line
    \newenvironment{Shaded}{}{}
    \newcommand{\KeywordTok}[1]{\textcolor[rgb]{0.00,0.44,0.13}{\textbf{{#1}}}}
    \newcommand{\DataTypeTok}[1]{\textcolor[rgb]{0.56,0.13,0.00}{{#1}}}
    \newcommand{\DecValTok}[1]{\textcolor[rgb]{0.25,0.63,0.44}{{#1}}}
    \newcommand{\BaseNTok}[1]{\textcolor[rgb]{0.25,0.63,0.44}{{#1}}}
    \newcommand{\FloatTok}[1]{\textcolor[rgb]{0.25,0.63,0.44}{{#1}}}
    \newcommand{\CharTok}[1]{\textcolor[rgb]{0.25,0.44,0.63}{{#1}}}
    \newcommand{\StringTok}[1]{\textcolor[rgb]{0.25,0.44,0.63}{{#1}}}
    \newcommand{\CommentTok}[1]{\textcolor[rgb]{0.38,0.63,0.69}{\textit{{#1}}}}
    \newcommand{\OtherTok}[1]{\textcolor[rgb]{0.00,0.44,0.13}{{#1}}}
    \newcommand{\AlertTok}[1]{\textcolor[rgb]{1.00,0.00,0.00}{\textbf{{#1}}}}
    \newcommand{\FunctionTok}[1]{\textcolor[rgb]{0.02,0.16,0.49}{{#1}}}
    \newcommand{\RegionMarkerTok}[1]{{#1}}
    \newcommand{\ErrorTok}[1]{\textcolor[rgb]{1.00,0.00,0.00}{\textbf{{#1}}}}
    \newcommand{\NormalTok}[1]{{#1}}
    
    % Additional commands for more recent versions of Pandoc
    \newcommand{\ConstantTok}[1]{\textcolor[rgb]{0.53,0.00,0.00}{{#1}}}
    \newcommand{\SpecialCharTok}[1]{\textcolor[rgb]{0.25,0.44,0.63}{{#1}}}
    \newcommand{\VerbatimStringTok}[1]{\textcolor[rgb]{0.25,0.44,0.63}{{#1}}}
    \newcommand{\SpecialStringTok}[1]{\textcolor[rgb]{0.73,0.40,0.53}{{#1}}}
    \newcommand{\ImportTok}[1]{{#1}}
    \newcommand{\DocumentationTok}[1]{\textcolor[rgb]{0.73,0.13,0.13}{\textit{{#1}}}}
    \newcommand{\AnnotationTok}[1]{\textcolor[rgb]{0.38,0.63,0.69}{\textbf{\textit{{#1}}}}}
    \newcommand{\CommentVarTok}[1]{\textcolor[rgb]{0.38,0.63,0.69}{\textbf{\textit{{#1}}}}}
    \newcommand{\VariableTok}[1]{\textcolor[rgb]{0.10,0.09,0.49}{{#1}}}
    \newcommand{\ControlFlowTok}[1]{\textcolor[rgb]{0.00,0.44,0.13}{\textbf{{#1}}}}
    \newcommand{\OperatorTok}[1]{\textcolor[rgb]{0.40,0.40,0.40}{{#1}}}
    \newcommand{\BuiltInTok}[1]{{#1}}
    \newcommand{\ExtensionTok}[1]{{#1}}
    \newcommand{\PreprocessorTok}[1]{\textcolor[rgb]{0.74,0.48,0.00}{{#1}}}
    \newcommand{\AttributeTok}[1]{\textcolor[rgb]{0.49,0.56,0.16}{{#1}}}
    \newcommand{\InformationTok}[1]{\textcolor[rgb]{0.38,0.63,0.69}{\textbf{\textit{{#1}}}}}
    \newcommand{\WarningTok}[1]{\textcolor[rgb]{0.38,0.63,0.69}{\textbf{\textit{{#1}}}}}
    
    
    % Define a nice break command that doesn't care if a line doesn't already
    % exist.
    \def\br{\hspace*{\fill} \\* }
    % Math Jax compatability definitions
    \def\gt{>}
    \def\lt{<}
    % Document parameters
    \title{5\_linalg\_statistics}
    
    
    

    % Pygments definitions
    
\makeatletter
\def\PY@reset{\let\PY@it=\relax \let\PY@bf=\relax%
    \let\PY@ul=\relax \let\PY@tc=\relax%
    \let\PY@bc=\relax \let\PY@ff=\relax}
\def\PY@tok#1{\csname PY@tok@#1\endcsname}
\def\PY@toks#1+{\ifx\relax#1\empty\else%
    \PY@tok{#1}\expandafter\PY@toks\fi}
\def\PY@do#1{\PY@bc{\PY@tc{\PY@ul{%
    \PY@it{\PY@bf{\PY@ff{#1}}}}}}}
\def\PY#1#2{\PY@reset\PY@toks#1+\relax+\PY@do{#2}}

\expandafter\def\csname PY@tok@kd\endcsname{\let\PY@bf=\textbf\def\PY@tc##1{\textcolor[rgb]{0.00,0.50,0.00}{##1}}}
\expandafter\def\csname PY@tok@ne\endcsname{\let\PY@bf=\textbf\def\PY@tc##1{\textcolor[rgb]{0.82,0.25,0.23}{##1}}}
\expandafter\def\csname PY@tok@vc\endcsname{\def\PY@tc##1{\textcolor[rgb]{0.10,0.09,0.49}{##1}}}
\expandafter\def\csname PY@tok@o\endcsname{\def\PY@tc##1{\textcolor[rgb]{0.40,0.40,0.40}{##1}}}
\expandafter\def\csname PY@tok@ow\endcsname{\let\PY@bf=\textbf\def\PY@tc##1{\textcolor[rgb]{0.67,0.13,1.00}{##1}}}
\expandafter\def\csname PY@tok@gd\endcsname{\def\PY@tc##1{\textcolor[rgb]{0.63,0.00,0.00}{##1}}}
\expandafter\def\csname PY@tok@vg\endcsname{\def\PY@tc##1{\textcolor[rgb]{0.10,0.09,0.49}{##1}}}
\expandafter\def\csname PY@tok@ch\endcsname{\let\PY@it=\textit\def\PY@tc##1{\textcolor[rgb]{0.25,0.50,0.50}{##1}}}
\expandafter\def\csname PY@tok@cpf\endcsname{\let\PY@it=\textit\def\PY@tc##1{\textcolor[rgb]{0.25,0.50,0.50}{##1}}}
\expandafter\def\csname PY@tok@cs\endcsname{\let\PY@it=\textit\def\PY@tc##1{\textcolor[rgb]{0.25,0.50,0.50}{##1}}}
\expandafter\def\csname PY@tok@il\endcsname{\def\PY@tc##1{\textcolor[rgb]{0.40,0.40,0.40}{##1}}}
\expandafter\def\csname PY@tok@gr\endcsname{\def\PY@tc##1{\textcolor[rgb]{1.00,0.00,0.00}{##1}}}
\expandafter\def\csname PY@tok@mi\endcsname{\def\PY@tc##1{\textcolor[rgb]{0.40,0.40,0.40}{##1}}}
\expandafter\def\csname PY@tok@m\endcsname{\def\PY@tc##1{\textcolor[rgb]{0.40,0.40,0.40}{##1}}}
\expandafter\def\csname PY@tok@s\endcsname{\def\PY@tc##1{\textcolor[rgb]{0.73,0.13,0.13}{##1}}}
\expandafter\def\csname PY@tok@gp\endcsname{\let\PY@bf=\textbf\def\PY@tc##1{\textcolor[rgb]{0.00,0.00,0.50}{##1}}}
\expandafter\def\csname PY@tok@nb\endcsname{\def\PY@tc##1{\textcolor[rgb]{0.00,0.50,0.00}{##1}}}
\expandafter\def\csname PY@tok@nv\endcsname{\def\PY@tc##1{\textcolor[rgb]{0.10,0.09,0.49}{##1}}}
\expandafter\def\csname PY@tok@w\endcsname{\def\PY@tc##1{\textcolor[rgb]{0.73,0.73,0.73}{##1}}}
\expandafter\def\csname PY@tok@kp\endcsname{\def\PY@tc##1{\textcolor[rgb]{0.00,0.50,0.00}{##1}}}
\expandafter\def\csname PY@tok@sc\endcsname{\def\PY@tc##1{\textcolor[rgb]{0.73,0.13,0.13}{##1}}}
\expandafter\def\csname PY@tok@gs\endcsname{\let\PY@bf=\textbf}
\expandafter\def\csname PY@tok@mo\endcsname{\def\PY@tc##1{\textcolor[rgb]{0.40,0.40,0.40}{##1}}}
\expandafter\def\csname PY@tok@go\endcsname{\def\PY@tc##1{\textcolor[rgb]{0.53,0.53,0.53}{##1}}}
\expandafter\def\csname PY@tok@k\endcsname{\let\PY@bf=\textbf\def\PY@tc##1{\textcolor[rgb]{0.00,0.50,0.00}{##1}}}
\expandafter\def\csname PY@tok@gh\endcsname{\let\PY@bf=\textbf\def\PY@tc##1{\textcolor[rgb]{0.00,0.00,0.50}{##1}}}
\expandafter\def\csname PY@tok@kt\endcsname{\def\PY@tc##1{\textcolor[rgb]{0.69,0.00,0.25}{##1}}}
\expandafter\def\csname PY@tok@c\endcsname{\let\PY@it=\textit\def\PY@tc##1{\textcolor[rgb]{0.25,0.50,0.50}{##1}}}
\expandafter\def\csname PY@tok@se\endcsname{\let\PY@bf=\textbf\def\PY@tc##1{\textcolor[rgb]{0.73,0.40,0.13}{##1}}}
\expandafter\def\csname PY@tok@err\endcsname{\def\PY@bc##1{\setlength{\fboxsep}{0pt}\fcolorbox[rgb]{1.00,0.00,0.00}{1,1,1}{\strut ##1}}}
\expandafter\def\csname PY@tok@cm\endcsname{\let\PY@it=\textit\def\PY@tc##1{\textcolor[rgb]{0.25,0.50,0.50}{##1}}}
\expandafter\def\csname PY@tok@sd\endcsname{\let\PY@it=\textit\def\PY@tc##1{\textcolor[rgb]{0.73,0.13,0.13}{##1}}}
\expandafter\def\csname PY@tok@sb\endcsname{\def\PY@tc##1{\textcolor[rgb]{0.73,0.13,0.13}{##1}}}
\expandafter\def\csname PY@tok@nd\endcsname{\def\PY@tc##1{\textcolor[rgb]{0.67,0.13,1.00}{##1}}}
\expandafter\def\csname PY@tok@s1\endcsname{\def\PY@tc##1{\textcolor[rgb]{0.73,0.13,0.13}{##1}}}
\expandafter\def\csname PY@tok@sr\endcsname{\def\PY@tc##1{\textcolor[rgb]{0.73,0.40,0.53}{##1}}}
\expandafter\def\csname PY@tok@kr\endcsname{\let\PY@bf=\textbf\def\PY@tc##1{\textcolor[rgb]{0.00,0.50,0.00}{##1}}}
\expandafter\def\csname PY@tok@sh\endcsname{\def\PY@tc##1{\textcolor[rgb]{0.73,0.13,0.13}{##1}}}
\expandafter\def\csname PY@tok@gi\endcsname{\def\PY@tc##1{\textcolor[rgb]{0.00,0.63,0.00}{##1}}}
\expandafter\def\csname PY@tok@s2\endcsname{\def\PY@tc##1{\textcolor[rgb]{0.73,0.13,0.13}{##1}}}
\expandafter\def\csname PY@tok@ge\endcsname{\let\PY@it=\textit}
\expandafter\def\csname PY@tok@sx\endcsname{\def\PY@tc##1{\textcolor[rgb]{0.00,0.50,0.00}{##1}}}
\expandafter\def\csname PY@tok@nn\endcsname{\let\PY@bf=\textbf\def\PY@tc##1{\textcolor[rgb]{0.00,0.00,1.00}{##1}}}
\expandafter\def\csname PY@tok@nc\endcsname{\let\PY@bf=\textbf\def\PY@tc##1{\textcolor[rgb]{0.00,0.00,1.00}{##1}}}
\expandafter\def\csname PY@tok@kc\endcsname{\let\PY@bf=\textbf\def\PY@tc##1{\textcolor[rgb]{0.00,0.50,0.00}{##1}}}
\expandafter\def\csname PY@tok@mb\endcsname{\def\PY@tc##1{\textcolor[rgb]{0.40,0.40,0.40}{##1}}}
\expandafter\def\csname PY@tok@nf\endcsname{\def\PY@tc##1{\textcolor[rgb]{0.00,0.00,1.00}{##1}}}
\expandafter\def\csname PY@tok@bp\endcsname{\def\PY@tc##1{\textcolor[rgb]{0.00,0.50,0.00}{##1}}}
\expandafter\def\csname PY@tok@vi\endcsname{\def\PY@tc##1{\textcolor[rgb]{0.10,0.09,0.49}{##1}}}
\expandafter\def\csname PY@tok@kn\endcsname{\let\PY@bf=\textbf\def\PY@tc##1{\textcolor[rgb]{0.00,0.50,0.00}{##1}}}
\expandafter\def\csname PY@tok@gu\endcsname{\let\PY@bf=\textbf\def\PY@tc##1{\textcolor[rgb]{0.50,0.00,0.50}{##1}}}
\expandafter\def\csname PY@tok@si\endcsname{\let\PY@bf=\textbf\def\PY@tc##1{\textcolor[rgb]{0.73,0.40,0.53}{##1}}}
\expandafter\def\csname PY@tok@na\endcsname{\def\PY@tc##1{\textcolor[rgb]{0.49,0.56,0.16}{##1}}}
\expandafter\def\csname PY@tok@mh\endcsname{\def\PY@tc##1{\textcolor[rgb]{0.40,0.40,0.40}{##1}}}
\expandafter\def\csname PY@tok@c1\endcsname{\let\PY@it=\textit\def\PY@tc##1{\textcolor[rgb]{0.25,0.50,0.50}{##1}}}
\expandafter\def\csname PY@tok@cp\endcsname{\def\PY@tc##1{\textcolor[rgb]{0.74,0.48,0.00}{##1}}}
\expandafter\def\csname PY@tok@mf\endcsname{\def\PY@tc##1{\textcolor[rgb]{0.40,0.40,0.40}{##1}}}
\expandafter\def\csname PY@tok@ni\endcsname{\let\PY@bf=\textbf\def\PY@tc##1{\textcolor[rgb]{0.60,0.60,0.60}{##1}}}
\expandafter\def\csname PY@tok@nl\endcsname{\def\PY@tc##1{\textcolor[rgb]{0.63,0.63,0.00}{##1}}}
\expandafter\def\csname PY@tok@nt\endcsname{\let\PY@bf=\textbf\def\PY@tc##1{\textcolor[rgb]{0.00,0.50,0.00}{##1}}}
\expandafter\def\csname PY@tok@no\endcsname{\def\PY@tc##1{\textcolor[rgb]{0.53,0.00,0.00}{##1}}}
\expandafter\def\csname PY@tok@gt\endcsname{\def\PY@tc##1{\textcolor[rgb]{0.00,0.27,0.87}{##1}}}
\expandafter\def\csname PY@tok@ss\endcsname{\def\PY@tc##1{\textcolor[rgb]{0.10,0.09,0.49}{##1}}}

\def\PYZbs{\char`\\}
\def\PYZus{\char`\_}
\def\PYZob{\char`\{}
\def\PYZcb{\char`\}}
\def\PYZca{\char`\^}
\def\PYZam{\char`\&}
\def\PYZlt{\char`\<}
\def\PYZgt{\char`\>}
\def\PYZsh{\char`\#}
\def\PYZpc{\char`\%}
\def\PYZdl{\char`\$}
\def\PYZhy{\char`\-}
\def\PYZsq{\char`\'}
\def\PYZdq{\char`\"}
\def\PYZti{\char`\~}
% for compatibility with earlier versions
\def\PYZat{@}
\def\PYZlb{[}
\def\PYZrb{]}
\makeatother


    % Exact colors from NB
    \definecolor{incolor}{rgb}{0.0, 0.0, 0.5}
    \definecolor{outcolor}{rgb}{0.545, 0.0, 0.0}



    
    % Prevent overflowing lines due to hard-to-break entities
    \sloppy 
    % Setup hyperref package
    \hypersetup{
      breaklinks=true,  % so long urls are correctly broken across lines
      colorlinks=true,
      urlcolor=urlcolor,
      linkcolor=linkcolor,
      citecolor=citecolor,
      }
    % Slightly bigger margins than the latex defaults
    
    \geometry{verbose,tmargin=1in,bmargin=1in,lmargin=1in,rmargin=1in}
    
    

    \begin{document}
    
    
    \maketitle
    
    

    
    \section{Introduction to Data Science with
Julia}\label{introduction-to-data-science-with-julia}

\section{目次}\label{ux76eeux6b21}

\begin{itemize}
\tightlist
\item
  \protect\hyperlink{ux7ddaux5f62ux4ee3ux6570}{線形代数}
\item
  \protect\hyperlink{ux7d71ux8a08ux91cfux306eux8a08ux7b97}{統計量の計算}
\item
  \protect\hyperlink{ux56deux5e30ux76f4ux7dda}{回帰直線}
\item
  \protect\hyperlink{ux30a2ux30f3ux30b9ux30b3ux30e0ux306eux4f8b}{アンスコムの例}
\item
  \protect\hyperlink{ux30c7ux30fcux30bfux5206ux6790ux5165ux9580}{データ分析入門}
\item
  \protect\hyperlink{ux7df4ux7fd2ux554fux984c}{練習問題}
\end{itemize}

    \section{線形代数}\label{ux7ddaux5f62ux4ee3ux6570}

Julia では線形代数の計算が標準機能として備わっています。

行列 \(A\) と行列 \(B\) の掛け算は

\begin{Shaded}
\begin{Highlighting}[]
\NormalTok{A * B}
\end{Highlighting}
\end{Shaded}

と書くだけです。

    \begin{Verbatim}[commandchars=\\\{\}]
{\color{incolor}In [{\color{incolor}1}]:} \PY{n}{A} \PY{o}{=} \PY{n}{reshape}\PY{p}{(}\PY{l+m+mi}{1}\PY{p}{:}\PY{l+m+mi}{9}\PY{p}{,} \PY{l+m+mi}{3}\PY{p}{,} \PY{l+m+mi}{3}\PY{p}{)}
        \PY{n}{B} \PY{o}{=} \PY{n}{rand}\PY{p}{(}\PY{l+m+mi}{1}\PY{p}{:}\PY{l+m+mi}{10}\PY{p}{,} \PY{l+m+mi}{3}\PY{p}{,} \PY{l+m+mi}{3}\PY{p}{)}
        \PY{p}{@}\PY{n}{show} \PY{n}{A}
        \PY{p}{@}\PY{n}{show} \PY{n}{B}
        \PY{p}{@}\PY{n}{show} \PY{n}{A} \PY{o}{*} \PY{n}{B}
\end{Verbatim}

    \begin{Verbatim}[commandchars=\\\{\}]
A = [1 4 7; 2 5 8; 3 6 9]
B = [3 6 2; 5 5 1; 1 6 3]
A * B = [30 68 27; 39 85 33; 48 102 39]

    \end{Verbatim}

            \begin{Verbatim}[commandchars=\\\{\}]
{\color{outcolor}Out[{\color{outcolor}1}]:} 3×3 Array\{Int64,2\}:
         30   68  27
         39   85  33
         48  102  39
\end{Verbatim}
        
    逆行列や行列式、固有値、固有ベクトルなども簡単に計算することが出来ます。

    \begin{Verbatim}[commandchars=\\\{\}]
{\color{incolor}In [{\color{incolor}2}]:} \PY{p}{@}\PY{n}{show} \PY{n}{C} \PY{o}{=} \PY{n}{rand}\PY{p}{(}\PY{l+m+mi}{3}\PY{p}{,} \PY{l+m+mi}{3}\PY{p}{)}
        \PY{n}{inv}\PY{p}{(}\PY{n}{C}\PY{p}{)} \PY{c}{\PYZsh{} 逆行列}
\end{Verbatim}

    \begin{Verbatim}[commandchars=\\\{\}]
C = rand(3,3) = [0.0316126 0.605445 0.0169347; 0.222539 0.544538 0.875881; 0.725021 0.297099 0.0125575]

    \end{Verbatim}

            \begin{Verbatim}[commandchars=\\\{\}]
{\color{outcolor}Out[{\color{outcolor}2}]:} 3×3 Array\{Float64,2\}:
         -0.686293  -0.00696504   1.41133  
          1.71241   -0.0321798   -0.0647877
         -0.890242   1.16348     -0.318304 
\end{Verbatim}
        
    \begin{Verbatim}[commandchars=\\\{\}]
{\color{incolor}In [{\color{incolor}3}]:} \PY{n}{det}\PY{p}{(}\PY{n}{C}\PY{p}{)} \PY{c}{\PYZsh{} 行列式}
\end{Verbatim}

            \begin{Verbatim}[commandchars=\\\{\}]
{\color{outcolor}Out[{\color{outcolor}3}]:} 0.3692088373360544
\end{Verbatim}
        
    \begin{Verbatim}[commandchars=\\\{\}]
{\color{incolor}In [{\color{incolor}4}]:} \PY{n}{eigvals}\PY{p}{(}\PY{n}{C}\PY{p}{)} \PY{c}{\PYZsh{} 固有値}
\end{Verbatim}

            \begin{Verbatim}[commandchars=\\\{\}]
{\color{outcolor}Out[{\color{outcolor}4}]:} 3-element Array\{Complex\{Float64\},1\}:
           1.17899+0.0im     
         -0.295143+0.475444im
         -0.295143-0.475444im
\end{Verbatim}
        
    \begin{Verbatim}[commandchars=\\\{\}]
{\color{incolor}In [{\color{incolor}5}]:} \PY{n}{eigvecs}\PY{p}{(}\PY{n}{C}\PY{p}{)} \PY{c}{\PYZsh{} 固有ベクトル}
\end{Verbatim}

            \begin{Verbatim}[commandchars=\\\{\}]
{\color{outcolor}Out[{\color{outcolor}5}]:} 3×3 Array\{Complex\{Float64\},2\}:
         -0.419653+0.0im  -0.0956797+0.556813im  -0.0956797-0.556813im
         -0.782415+0.0im     -0.4028-0.375644im     -0.4028+0.375644im
         -0.460129+0.0im    0.614368+0.0im         0.614368-0.0im     
\end{Verbatim}
        
    \begin{Verbatim}[commandchars=\\\{\}]
{\color{incolor}In [{\color{incolor}6}]:} \PY{n}{trace}\PY{p}{(}\PY{n}{C}\PY{p}{)} \PY{c}{\PYZsh{} トレース}
\end{Verbatim}

            \begin{Verbatim}[commandchars=\\\{\}]
{\color{outcolor}Out[{\color{outcolor}6}]:} 0.5887077924636039
\end{Verbatim}
        
    \begin{Verbatim}[commandchars=\\\{\}]
{\color{incolor}In [{\color{incolor}7}]:} \PY{n}{rank}\PY{p}{(}\PY{n}{C}\PY{p}{)} \PY{c}{\PYZsh{} ランク}
\end{Verbatim}

            \begin{Verbatim}[commandchars=\\\{\}]
{\color{outcolor}Out[{\color{outcolor}7}]:} 3
\end{Verbatim}
        
    \begin{Verbatim}[commandchars=\\\{\}]
{\color{incolor}In [{\color{incolor}8}]:} \PY{n}{x} \PY{o}{=} \PY{p}{[}\PY{l+m+mi}{1}\PY{p}{,} \PY{l+m+mi}{2}\PY{p}{,} \PY{l+m+mi}{3}\PY{p}{]}
        \PY{n}{y} \PY{o}{=} \PY{p}{[}\PY{l+m+mi}{4}\PY{p}{,} \PY{l+m+mi}{5}\PY{p}{,} \PY{l+m+mi}{6}\PY{p}{]}
        \PY{n}{dot}\PY{p}{(}\PY{n}{x}\PY{p}{,}\PY{n}{y}\PY{p}{)} \PY{c}{\PYZsh{} 内積}
\end{Verbatim}

            \begin{Verbatim}[commandchars=\\\{\}]
{\color{outcolor}Out[{\color{outcolor}8}]:} 32
\end{Verbatim}
        
    \begin{Verbatim}[commandchars=\\\{\}]
{\color{incolor}In [{\color{incolor} }]:} 
\end{Verbatim}

    線形方程式 \(Ax = y\) を解く場合には

\begin{Shaded}
\begin{Highlighting}[]
\NormalTok{A \textbackslash{} y }\CommentTok{# ¥ はバックスラッシュ}
\end{Highlighting}
\end{Shaded}

とします。行列 \(A\) の逆行列は inv(A) として求められるので

\begin{Shaded}
\begin{Highlighting}[]
\NormalTok{inv(A) * y}
\end{Highlighting}
\end{Shaded}

としても同様の結果が得られますが、A ~y の方がより計算精度が上がります。

    \begin{Verbatim}[commandchars=\\\{\}]
{\color{incolor}In [{\color{incolor}9}]:} \PY{n}{A} \PY{o}{=} \PY{n}{rand}\PY{p}{(}\PY{l+m+mi}{3}\PY{p}{,}\PY{l+m+mi}{3}\PY{p}{)}
        \PY{n}{y} \PY{o}{=} \PY{n}{rand}\PY{p}{(}\PY{l+m+mi}{3}\PY{p}{)}
        \PY{n}{A} \PYZbs{} \PY{n}{y}
\end{Verbatim}

            \begin{Verbatim}[commandchars=\\\{\}]
{\color{outcolor}Out[{\color{outcolor}9}]:} 3-element Array\{Float64,1\}:
         -0.309406
          0.761443
         -0.538079
\end{Verbatim}
        
    \begin{Verbatim}[commandchars=\\\{\}]
{\color{incolor}In [{\color{incolor} }]:} 
\end{Verbatim}

    その他の行列計算に関しては\href{http://docs.julialang.org/en/stable/manual/linear-algebra/}{公式ドキュメント}を読んでください。

    \protect\hyperlink{ux76eeux6b21}{目次に戻る}

    \section{統計量の計算}\label{ux7d71ux8a08ux91cfux306eux8a08ux7b97}

\subsection{量的データ (numerical
data)}\label{ux91cfux7684ux30c7ux30fcux30bf-numerical-data}

Julia では平均や分散などを計算する関数が標準で備わっています。

    \begin{Verbatim}[commandchars=\\\{\}]
{\color{incolor}In [{\color{incolor}10}]:} \PY{c}{\PYZsh{} 平均 (mean)}
         \PY{p}{@}\PY{n}{show} \PY{n}{x} \PY{o}{=} \PY{n}{rand}\PY{p}{(}\PY{l+m+mi}{5}\PY{p}{)}
         \PY{n}{mean}\PY{p}{(}\PY{n}{x}\PY{p}{)}
\end{Verbatim}

    \begin{Verbatim}[commandchars=\\\{\}]
x = rand(5) = [0.353628,0.903524,0.159742,0.810534,0.950088]

    \end{Verbatim}

            \begin{Verbatim}[commandchars=\\\{\}]
{\color{outcolor}Out[{\color{outcolor}10}]:} 0.6355032675516282
\end{Verbatim}
        
    \begin{Verbatim}[commandchars=\\\{\}]
{\color{incolor}In [{\color{incolor}11}]:} \PY{c}{\PYZsh{} 分散 (variance)}
         \PY{p}{@}\PY{n}{show} \PY{n}{x} \PY{o}{=} \PY{n}{rand}\PY{p}{(}\PY{l+m+mi}{5}\PY{p}{)}
         \PY{n}{var}\PY{p}{(}\PY{n}{x}\PY{p}{)} \PY{c}{\PYZsh{} 補正をなくす場合は、 var(x, corrected=false) とする}
\end{Verbatim}

    \begin{Verbatim}[commandchars=\\\{\}]
x = rand(5) = [0.885045,0.751102,0.742,0.48118,0.592068]

    \end{Verbatim}

            \begin{Verbatim}[commandchars=\\\{\}]
{\color{outcolor}Out[{\color{outcolor}11}]:} 0.024419059408298135
\end{Verbatim}
        
    \begin{Verbatim}[commandchars=\\\{\}]
{\color{incolor}In [{\color{incolor}12}]:} \PY{c}{\PYZsh{} 標準偏差 (standard deviation)}
         \PY{p}{@}\PY{n}{show} \PY{n}{x} \PY{o}{=} \PY{n}{rand}\PY{p}{(}\PY{l+m+mi}{5}\PY{p}{)}
         \PY{n}{std}\PY{p}{(}\PY{n}{x}\PY{p}{)} \PY{c}{\PYZsh{} 補正をなくす場合は、 std(x, corrected=false) とする}
\end{Verbatim}

    \begin{Verbatim}[commandchars=\\\{\}]
x = rand(5) = [0.101201,0.790782,0.429018,0.421794,0.422853]

    \end{Verbatim}

            \begin{Verbatim}[commandchars=\\\{\}]
{\color{outcolor}Out[{\color{outcolor}12}]:} 0.24410191862505906
\end{Verbatim}
        
    \begin{Verbatim}[commandchars=\\\{\}]
{\color{incolor}In [{\color{incolor} }]:} 
\end{Verbatim}

    \begin{Verbatim}[commandchars=\\\{\}]
{\color{incolor}In [{\color{incolor}13}]:} \PY{c}{\PYZsh{} 中央値 (median)}
         \PY{n}{x} \PY{o}{=} \PY{l+m+mi}{1}\PY{p}{:}\PY{l+m+mi}{5}
         \PY{n}{median}\PY{p}{(}\PY{n}{x}\PY{p}{)}
\end{Verbatim}

            \begin{Verbatim}[commandchars=\\\{\}]
{\color{outcolor}Out[{\color{outcolor}13}]:} 3.0
\end{Verbatim}
        
    \begin{Verbatim}[commandchars=\\\{\}]
{\color{incolor}In [{\color{incolor}14}]:} \PY{c}{\PYZsh{} 第1四分位点(lower quartile)}
         \PY{n}{quantile}\PY{p}{(}\PY{n}{x}\PY{p}{,} \PY{l+m+mi}{1}\PY{o}{/}\PY{l+m+mi}{4}\PY{p}{)}
\end{Verbatim}

            \begin{Verbatim}[commandchars=\\\{\}]
{\color{outcolor}Out[{\color{outcolor}14}]:} 2.0
\end{Verbatim}
        
    \begin{Verbatim}[commandchars=\\\{\}]
{\color{incolor}In [{\color{incolor}15}]:} \PY{c}{\PYZsh{} 第1四分位点, 中央値, 第3四分位点(upper quartile)}
         \PY{n}{quantile}\PY{p}{(}\PY{n}{x}\PY{p}{,} \PY{p}{[}\PY{l+m+mi}{1}\PY{o}{/}\PY{l+m+mi}{4}\PY{p}{,} \PY{l+m+mi}{1}\PY{o}{/}\PY{l+m+mi}{2}\PY{p}{,} \PY{l+m+mi}{3}\PY{o}{/}\PY{l+m+mi}{4}\PY{p}{]}\PY{p}{)}
\end{Verbatim}

            \begin{Verbatim}[commandchars=\\\{\}]
{\color{outcolor}Out[{\color{outcolor}15}]:} 3-element Array\{Float64,1\}:
          2.0
          3.0
          4.0
\end{Verbatim}
        
    \begin{Verbatim}[commandchars=\\\{\}]
{\color{incolor}In [{\color{incolor}16}]:} \PY{c}{\PYZsh{} 最大値}
         \PY{n}{maximum}\PY{p}{(}\PY{n}{x}\PY{p}{)}
\end{Verbatim}

            \begin{Verbatim}[commandchars=\\\{\}]
{\color{outcolor}Out[{\color{outcolor}16}]:} 5
\end{Verbatim}
        
    \begin{Verbatim}[commandchars=\\\{\}]
{\color{incolor}In [{\color{incolor}17}]:} \PY{c}{\PYZsh{} 最小値}
         \PY{n}{minimum}\PY{p}{(}\PY{n}{x}\PY{p}{)}
\end{Verbatim}

            \begin{Verbatim}[commandchars=\\\{\}]
{\color{outcolor}Out[{\color{outcolor}17}]:} 1
\end{Verbatim}
        
    \begin{Verbatim}[commandchars=\\\{\}]
{\color{incolor}In [{\color{incolor}18}]:} \PY{c}{\PYZsh{} 最小値と最大値を同時に計算}
         \PY{n}{extrema}\PY{p}{(}\PY{n}{x}\PY{p}{)}
\end{Verbatim}

            \begin{Verbatim}[commandchars=\\\{\}]
{\color{outcolor}Out[{\color{outcolor}18}]:} (1,5)
\end{Verbatim}
        
    \begin{Verbatim}[commandchars=\\\{\}]
{\color{incolor}In [{\color{incolor}19}]:} \PY{c}{\PYZsh{} 相関係数}
         \PY{n}{x} \PY{o}{=} \PY{l+m+mf}{1.0}\PY{p}{:}\PY{l+m+mf}{1.0}\PY{p}{:}\PY{l+m+mf}{12.0}
         \PY{n}{y} \PY{o}{=} \PY{n}{x} \PY{o}{.}\PY{o}{+} \PY{n}{randn}\PY{p}{(}\PY{n}{length}\PY{p}{(}\PY{n}{x}\PY{p}{)}\PY{p}{)}
         \PY{n}{cor}\PY{p}{(}\PY{n}{x}\PY{p}{,}\PY{n}{y}\PY{p}{)}
\end{Verbatim}

            \begin{Verbatim}[commandchars=\\\{\}]
{\color{outcolor}Out[{\color{outcolor}19}]:} 0.9835687891956586
\end{Verbatim}
        
    \begin{Verbatim}[commandchars=\\\{\}]
{\color{incolor}In [{\color{incolor} }]:} 
\end{Verbatim}

    mean や var などで
2次元配列を引数に取ると要素全体の平均などになりますが、第2引数に 1
と指定すると列ごとの平均、 2 を指定すると行ごとの平均になります。

    \begin{Verbatim}[commandchars=\\\{\}]
{\color{incolor}In [{\color{incolor}20}]:} \PY{n}{x} \PY{o}{=} \PY{p}{[}\PY{l+m+mi}{1} \PY{l+m+mi}{2} \PY{l+m+mi}{3}
             \PY{l+m+mi}{4} \PY{l+m+mi}{5} \PY{l+m+mi}{6} 
             \PY{l+m+mi}{7} \PY{l+m+mi}{8} \PY{l+m+mi}{9}\PY{p}{]}
         \PY{p}{@}\PY{n}{show} \PY{n}{mean}\PY{p}{(}\PY{n}{x}\PY{p}{)}
         \PY{p}{@}\PY{n}{show} \PY{n}{mean}\PY{p}{(}\PY{n}{x}\PY{p}{,} \PY{l+m+mi}{1}\PY{p}{)} \PY{c}{\PYZsh{} 列ごとの平均}
         \PY{p}{@}\PY{n}{show} \PY{n}{mean}\PY{p}{(}\PY{n}{x}\PY{p}{,} \PY{l+m+mi}{2}\PY{p}{)} \PY{c}{\PYZsh{} 行ごとの平均}
\end{Verbatim}

    \begin{Verbatim}[commandchars=\\\{\}]
mean(x) = 5.0
mean(x,1) = [4.0 5.0 6.0]
mean(x,2) = [2.0; 5.0; 8.0]

    \end{Verbatim}

    \begin{Verbatim}[commandchars=\\\{\}]
{\color{incolor}In [{\color{incolor} }]:} 
\end{Verbatim}

    \protect\hyperlink{ux76eeux6b21}{目次に戻る}

    \href{http://statsbasejl.readthedocs.io/en/latest/index.html}{StatsBase}
パッケージを使うと統計量の計算がさらにやりやすくなります。

    \begin{Verbatim}[commandchars=\\\{\}]
{\color{incolor}In [{\color{incolor}21}]:} \PY{k}{using} \PY{n}{StatsBase}
\end{Verbatim}

    summarystats
を使うと、平均と五数要約(\href{https://en.wikipedia.org/wiki/Five-number_summary}{five-number
summary})を一度に計算できます。

    \begin{Verbatim}[commandchars=\\\{\}]
{\color{incolor}In [{\color{incolor}22}]:} \PY{p}{@}\PY{n}{show} \PY{n}{x} \PY{o}{=} \PY{n}{rand}\PY{p}{(}\PY{l+m+mi}{10}\PY{p}{)}
         \PY{n}{result} \PY{o}{=} \PY{n}{StatsBase}\PY{o}{.}\PY{n}{summarystats}\PY{p}{(}\PY{n}{x}\PY{p}{)}
\end{Verbatim}

    \begin{Verbatim}[commandchars=\\\{\}]
x = rand(10) = [0.849495,0.102439,0.979398,0.132961,0.448198,0.801706,0.613595,0.511562,0.641375,0.295273]

    \end{Verbatim}

    \begin{Verbatim}[commandchars=\\\{\}]
{\color{incolor}In [{\color{incolor}23}]:} \PY{n+nb}{typeof}\PY{p}{(}\PY{n}{result}\PY{p}{)}
\end{Verbatim}

            \begin{Verbatim}[commandchars=\\\{\}]
{\color{outcolor}Out[{\color{outcolor}23}]:} StatsBase.SummaryStats\{Float64\}
\end{Verbatim}
        
    \begin{Verbatim}[commandchars=\\\{\}]
{\color{incolor}In [{\color{incolor}24}]:} \PY{n}{fieldnames}\PY{p}{(}\PY{n}{result}\PY{p}{)} \PY{c}{\PYZsh{} 各データへアクセスするための名前}
\end{Verbatim}

            \begin{Verbatim}[commandchars=\\\{\}]
{\color{outcolor}Out[{\color{outcolor}24}]:} 6-element Array\{Symbol,1\}:
          :mean  
          :min   
          :q25   
          :median
          :q75   
          :max   
\end{Verbatim}
        
    \begin{Verbatim}[commandchars=\\\{\}]
{\color{incolor}In [{\color{incolor}25}]:} \PY{n}{result}\PY{o}{.}\PY{n}{mean}
\end{Verbatim}

            \begin{Verbatim}[commandchars=\\\{\}]
{\color{outcolor}Out[{\color{outcolor}25}]:} 0.5376001014419926
\end{Verbatim}
        
    \begin{Verbatim}[commandchars=\\\{\}]
{\color{incolor}In [{\color{incolor}26}]:} \PY{n}{result}\PY{o}{.}\PY{n}{q25}
\end{Verbatim}

            \begin{Verbatim}[commandchars=\\\{\}]
{\color{outcolor}Out[{\color{outcolor}26}]:} 0.33350458271318084
\end{Verbatim}
        
    四分位範囲 (Interquartile Range, IQR) を計算するには iqr を使います。

    \begin{Verbatim}[commandchars=\\\{\}]
{\color{incolor}In [{\color{incolor}27}]:} \PY{n}{StatsBase}\PY{o}{.}\PY{n}{iqr}\PY{p}{(}\PY{n}{x}\PY{p}{)}
\end{Verbatim}

            \begin{Verbatim}[commandchars=\\\{\}]
{\color{outcolor}Out[{\color{outcolor}27}]:} 0.42811837334235747
\end{Verbatim}
        
    \href{https://en.wikipedia.org/wiki/Standard_score}{Z-score}
の計算も出来ます。

    \begin{Verbatim}[commandchars=\\\{\}]
{\color{incolor}In [{\color{incolor}28}]:} \PY{n}{StatsBase}\PY{o}{.}\PY{n}{zscore}\PY{p}{(}\PY{n}{x}\PY{p}{)}
\end{Verbatim}

            \begin{Verbatim}[commandchars=\\\{\}]
{\color{outcolor}Out[{\color{outcolor}28}]:} 10-element Array\{Float64,1\}:
           1.04662  
          -1.46026  
           1.48253  
          -1.35784  
          -0.300004 
           0.886252 
           0.255013 
          -0.0873765
           0.348236 
          -0.81317  
\end{Verbatim}
        
    \begin{Verbatim}[commandchars=\\\{\}]
{\color{incolor}In [{\color{incolor} }]:} 
\end{Verbatim}

    \protect\hyperlink{ux76eeux6b21}{目次に戻る}

    \subsection{質的データ (categorical
data)}\label{ux8ceaux7684ux30c7ux30fcux30bf-categorical-data}

質的データを調べるときには countmap や proportionmap を使うと便利です。
共通要素が何個(何割)あるのかがわかります。

返り値は各要素を key とする辞書です。

    \begin{Verbatim}[commandchars=\\\{\}]
{\color{incolor}In [{\color{incolor}29}]:} \PY{p}{@}\PY{n}{show} \PY{n}{x} \PY{o}{=} \PY{n}{rand}\PY{p}{(}\PY{l+s+sc}{\PYZsq{}a\PYZsq{}}\PY{p}{:}\PY{l+s+sc}{\PYZsq{}c\PYZsq{}}\PY{p}{,} \PY{l+m+mi}{10}\PY{p}{)}
         \PY{n}{StatsBase}\PY{o}{.}\PY{n}{countmap}\PY{p}{(}\PY{n}{x}\PY{p}{)} \PY{c}{\PYZsh{} 共通要素の個数}
\end{Verbatim}

    \begin{Verbatim}[commandchars=\\\{\}]
x = rand('a':'c',10) = ['a','a','b','b','b','a','b','b','b','b']

    \end{Verbatim}

            \begin{Verbatim}[commandchars=\\\{\}]
{\color{outcolor}Out[{\color{outcolor}29}]:} Dict\{Char,Int64\} with 2 entries:
           'b' => 7
           'a' => 3
\end{Verbatim}
        
    \begin{Verbatim}[commandchars=\\\{\}]
{\color{incolor}In [{\color{incolor}30}]:} \PY{n}{abc} \PY{o}{=} \PY{n}{StatsBase}\PY{o}{.}\PY{n}{proportionmap}\PY{p}{(}\PY{n}{x}\PY{p}{)} \PY{c}{\PYZsh{} 共通要素の全体の割合}
\end{Verbatim}

            \begin{Verbatim}[commandchars=\\\{\}]
{\color{outcolor}Out[{\color{outcolor}30}]:} Dict\{Char,Float64\} with 2 entries:
           'b' => 0.7
           'a' => 0.3
\end{Verbatim}
        
    \begin{Verbatim}[commandchars=\\\{\}]
{\color{incolor}In [{\color{incolor}31}]:} \PY{n}{abc}\PY{p}{[}\PY{l+s+sc}{\PYZsq{}a\PYZsq{}}\PY{p}{]}
\end{Verbatim}

            \begin{Verbatim}[commandchars=\\\{\}]
{\color{outcolor}Out[{\color{outcolor}31}]:} 0.3
\end{Verbatim}
        
    \protect\hyperlink{ux76eeux6b21}{目次に戻る}

    \subsection{度数分布}\label{ux5ea6ux6570ux5206ux5e03}

まずは Plots を使ってヒストグラムを書いてみましょう。

    \begin{Verbatim}[commandchars=\\\{\}]
{\color{incolor}In [{\color{incolor}32}]:} \PY{k}{import} \PY{n}{Plots}
         \PY{n}{Plots}\PY{o}{.}\PY{n}{gr}\PY{p}{(}\PY{n}{leg}\PY{o}{=}\PY{n}{false}\PY{p}{)}
\end{Verbatim}

            \begin{Verbatim}[commandchars=\\\{\}]
{\color{outcolor}Out[{\color{outcolor}32}]:} Plots.GRBackend()
\end{Verbatim}
        
    \begin{Verbatim}[commandchars=\\\{\}]
{\color{incolor}In [{\color{incolor}33}]:} \PY{n}{nd} \PY{o}{=} \PY{n}{randn}\PY{p}{(}\PY{l+m+mi}{1000}\PY{p}{)} \PY{c}{\PYZsh{} 正規分布に従う乱数 1000点}
         \PY{n}{Plots}\PY{o}{.}\PY{n}{histogram}\PY{p}{(}\PY{n}{nd}\PY{p}{)}
\end{Verbatim}

    bin を変える場合は

\begin{Shaded}
\begin{Highlighting}[]
\NormalTok{nbins = }\FloatTok{20}
\end{Highlighting}
\end{Shaded}

や

\begin{Shaded}
\begin{Highlighting}[]
\NormalTok{nbins = -}\FloatTok{5}\NormalTok{:}\FloatTok{0.5}\NormalTok{:}\FloatTok{5}
\end{Highlighting}
\end{Shaded}

などとします。分割数を指定する場合は前者で、分割幅を決めたい場合は後者を使うと便利です。

    \begin{Verbatim}[commandchars=\\\{\}]
{\color{incolor}In [{\color{incolor}34}]:} \PY{n}{Plots}\PY{o}{.}\PY{n}{histogram}\PY{p}{(}\PY{n}{nd}\PY{p}{,} \PY{n}{nbins} \PY{o}{=} \PY{l+m+mi}{20}\PY{p}{)} \PY{c}{\PYZsh{} 20分割}
\end{Verbatim}

    \begin{Verbatim}[commandchars=\\\{\}]
{\color{incolor}In [{\color{incolor}35}]:} \PY{n}{Plots}\PY{o}{.}\PY{n}{histogram}\PY{p}{(}\PY{n}{nd}\PY{p}{,} \PY{n}{nbins} \PY{o}{=} \PY{o}{\PYZhy{}}\PY{l+m+mf}{5.0}\PY{p}{:}\PY{l+m+mf}{0.5}\PY{p}{:}\PY{l+m+mf}{5.0}\PY{p}{)} \PY{c}{\PYZsh{} bin幅を 0.5 にし、\PYZhy{}5 〜 5 の範囲でプロット}
\end{Verbatim}

    \begin{Verbatim}[commandchars=\\\{\}]
{\color{incolor}In [{\color{incolor}36}]:} \PY{n}{Plots}\PY{o}{.}\PY{n}{histogram}\PY{p}{(}\PY{n}{nd}\PY{p}{,} \PY{n}{nbins} \PY{o}{=} \PY{o}{\PYZhy{}}\PY{l+m+mf}{5.0}\PY{p}{:}\PY{l+m+mf}{0.5}\PY{p}{:}\PY{l+m+mf}{5.0}\PY{p}{,} \PY{n}{norm}\PY{o}{=}\PY{n}{true}\PY{p}{)} \PY{c}{\PYZsh{} 正規化}
\end{Verbatim}

    Plots
を使ってヒストグラムを書けばどのような分布なのかということはわかりますが、各
bin の中に何サンプルあるのかという具体的な数値はわかりません。

具体的な数値を知りたい場合は

\begin{Shaded}
\begin{Highlighting}[]
    \NormalTok{fit(Histogram, nd, nbins = }\FloatTok{20}\NormalTok{)}
    \NormalTok{or}
    \NormalTok{fit(Histogram, nd, -}\FloatTok{5.0}\NormalTok{:}\FloatTok{0.5}\NormalTok{:}\FloatTok{5.0}\NormalTok{)}
\end{Highlighting}
\end{Shaded}

などとします。

結果は

\begin{Shaded}
\begin{Highlighting}[]
\NormalTok{StatsBase.Histogram\{}\DataTypeTok{Int64}\NormalTok{,}\FloatTok{1}\NormalTok{,}\DataTypeTok{Tuple}\NormalTok{\{FloatRange\{}\DataTypeTok{Float64}\NormalTok{\}\}\}}
\NormalTok{edges:}
  \NormalTok{-}\FloatTok{3.5}\NormalTok{:}\FloatTok{0.5}\NormalTok{:}\FloatTok{3.0}
\NormalTok{weights: [}\FloatTok{1}\NormalTok{,}\FloatTok{2}\NormalTok{,}\FloatTok{13}\NormalTok{,}\FloatTok{46}\NormalTok{,}\FloatTok{84}\NormalTok{,}\FloatTok{149}\NormalTok{,}\FloatTok{173}\NormalTok{,}\FloatTok{206}\NormalTok{,}\FloatTok{164}\NormalTok{,}\FloatTok{83}\NormalTok{,}\FloatTok{46}\NormalTok{,}\FloatTok{27}\NormalTok{,}\FloatTok{6}\NormalTok{]}
\NormalTok{closed: right}
\end{Highlighting}
\end{Shaded}

のようになります。ここで edges は範囲と分割幅、weights
が度数を表します。 各値は 変数名.edges や 変数名.weights
をすることで抜き出すことが出来ます。

    \begin{Verbatim}[commandchars=\\\{\}]
{\color{incolor}In [{\color{incolor}37}]:} \PY{n}{ndfreq1} \PY{o}{=} \PY{n}{fit}\PY{p}{(}\PY{n}{Histogram}\PY{p}{,} \PY{n}{nd}\PY{p}{,} \PY{n}{nbins} \PY{o}{=} \PY{l+m+mi}{20}\PY{p}{)} \PY{c}{\PYZsh{} 分割数を指定}
\end{Verbatim}

            \begin{Verbatim}[commandchars=\\\{\}]
{\color{outcolor}Out[{\color{outcolor}37}]:} StatsBase.Histogram\{Int64,1,Tuple\{FloatRange\{Float64\}\}\}
         edges:
           -3.5:0.5:3.5
         weights: [3,6,12,43,79,166,188,179,162,80,50,26,5,1]
         closed: right
\end{Verbatim}
        
    \begin{Verbatim}[commandchars=\\\{\}]
{\color{incolor}In [{\color{incolor}38}]:} \PY{n}{ndfreq2} \PY{o}{=} \PY{n}{fit}\PY{p}{(}\PY{n}{Histogram}\PY{p}{,} \PY{n}{nd}\PY{p}{,} \PY{o}{\PYZhy{}}\PY{l+m+mf}{5.0}\PY{p}{:}\PY{l+m+mf}{0.5}\PY{p}{:}\PY{l+m+mf}{5.0}\PY{p}{)} \PY{c}{\PYZsh{} 幅と範囲を指定}
\end{Verbatim}

            \begin{Verbatim}[commandchars=\\\{\}]
{\color{outcolor}Out[{\color{outcolor}38}]:} StatsBase.Histogram\{Int64,1,Tuple\{FloatRange\{Float64\}\}\}
         edges:
           -5.0:0.5:5.0
         weights: [0,0,0,3,6,12,43,79,166,188,179,162,80,50,26,5,1,0,0,0]
         closed: right
\end{Verbatim}
        
    \begin{Verbatim}[commandchars=\\\{\}]
{\color{incolor}In [{\color{incolor}39}]:} \PY{n}{ndfreq2}\PY{o}{.}\PY{n}{edges}
\end{Verbatim}

            \begin{Verbatim}[commandchars=\\\{\}]
{\color{outcolor}Out[{\color{outcolor}39}]:} (-5.0:0.5:5.0,)
\end{Verbatim}
        
    \begin{Verbatim}[commandchars=\\\{\}]
{\color{incolor}In [{\color{incolor}40}]:} \PY{n}{ndfreq2}\PY{o}{.}\PY{n}{weights}
\end{Verbatim}

            \begin{Verbatim}[commandchars=\\\{\}]
{\color{outcolor}Out[{\color{outcolor}40}]:} 20-element Array\{Int64,1\}:
            0
            0
            0
            3
            6
           12
           43
           79
          166
          188
          179
          162
           80
           50
           26
            5
            1
            0
            0
            0
\end{Verbatim}
        
    求めた結果をヒストグラムと一緒に描写してみます。

    \begin{Verbatim}[commandchars=\\\{\}]
{\color{incolor}In [{\color{incolor}41}]:} \PY{n}{Plots}\PY{o}{.}\PY{n}{histogram}\PY{p}{(}\PY{n}{nd}\PY{p}{,} \PY{n}{nbins} \PY{o}{=} \PY{o}{\PYZhy{}}\PY{l+m+mf}{5.0}\PY{p}{:}\PY{l+m+mf}{0.5}\PY{p}{:}\PY{l+m+mf}{5.0}\PY{p}{)}
         \PY{n}{x} \PY{o}{=} \PY{p}{[}\PY{p}{(}\PY{n}{ndfreq2}\PY{o}{.}\PY{n}{edges}\PY{p}{[}\PY{l+m+mi}{1}\PY{p}{]}\PY{p}{[}\PY{n}{i}\PY{p}{]} \PY{o}{+} \PY{n}{ndfreq2}\PY{o}{.}\PY{n}{edges}\PY{p}{[}\PY{l+m+mi}{1}\PY{p}{]}\PY{p}{[}\PY{n}{i}\PY{o}{+}\PY{l+m+mi}{1}\PY{p}{]}\PY{p}{)}\PY{o}{/}\PY{l+m+mi}{2} \PY{k}{for} \PY{n}{i} \PY{k}{in} \PY{l+m+mi}{1}\PY{p}{:}\PY{n}{length}\PY{p}{(}\PY{n}{ndfreq2}\PY{o}{.}\PY{n}{edges}\PY{p}{[}\PY{l+m+mi}{1}\PY{p}{]}\PY{p}{)}\PY{o}{\PYZhy{}}\PY{l+m+mi}{1}\PY{p}{]} \PY{c}{\PYZsh{} }
         \PY{n}{Plots}\PY{o}{.}\PY{n}{plot!}\PY{p}{(}\PY{n}{x}\PY{p}{,} \PY{n}{ndfreq2}\PY{o}{.}\PY{n}{weights}\PY{p}{,} \PY{n}{marker}\PY{o}{=}\PY{p}{:}\PY{n}{circle}\PY{p}{)}
\end{Verbatim}

    \begin{Verbatim}[commandchars=\\\{\}]
{\color{incolor}In [{\color{incolor} }]:} 
\end{Verbatim}

    最頻値 (mode) を求める場合は mode を使います。

    \begin{Verbatim}[commandchars=\\\{\}]
{\color{incolor}In [{\color{incolor}42}]:} \PY{p}{@}\PY{n}{show} \PY{n}{a} \PY{o}{=} \PY{n}{rand}\PY{p}{(}\PY{l+s+sc}{\PYZsq{}A\PYZsq{}}\PY{p}{:}\PY{l+s+sc}{\PYZsq{}C\PYZsq{}}\PY{p}{,} \PY{l+m+mi}{10}\PY{p}{)}
         \PY{n}{StatsBase}\PY{o}{.}\PY{n}{mode}\PY{p}{(}\PY{n}{a}\PY{p}{)}
\end{Verbatim}

    \begin{Verbatim}[commandchars=\\\{\}]
a = rand('A':'C',10) = ['A','B','A','B','A','A','A','A','C','C']

    \end{Verbatim}

            \begin{Verbatim}[commandchars=\\\{\}]
{\color{outcolor}Out[{\color{outcolor}42}]:} 'A'
\end{Verbatim}
        
    \begin{Verbatim}[commandchars=\\\{\}]
{\color{incolor}In [{\color{incolor} }]:} 
\end{Verbatim}

    \protect\hyperlink{ux76eeux6b21}{目次に戻る}

    \section{回帰直線}\label{ux56deux5e30ux76f4ux7dda}

Julia で回帰直線の切片、傾きを求めるには linreg 関数を使います。

    \begin{Verbatim}[commandchars=\\\{\}]
{\color{incolor}In [{\color{incolor}43}]:} \PY{o}{?}\PY{n}{linreg}
\end{Verbatim}

    \begin{Verbatim}[commandchars=\\\{\}]
search: \textbf{l}\textbf{i}\textbf{n}\textbf{r}\textbf{e}\textbf{g} \textbf{l}\textbf{i}\textbf{n}ea\textbf{r}indic\textbf{e}s \textbf{L}\textbf{i}\textbf{n}eNumbe\textbf{r}Nod\textbf{e}


    \end{Verbatim}
\texttt{\color{outcolor}Out[{\color{outcolor}43}]:}
    
    \begin{verbatim}
linreg(x, y)
\end{verbatim}

Perform simple linear regression using Ordinary Least Squares. Returns
\texttt{a} and \texttt{b} such that \texttt{a\ +\ b*x} is the closest
straight line to the given points \texttt{(x,\ y)}, i.e., such that the
squared error between \texttt{y} and \texttt{a\ +\ b*x} is minimized.

Examples:

\begin{verbatim}
using PyPlot
x = 1.0:12.0
y = [5.5, 6.3, 7.6, 8.8, 10.9, 11.79, 13.48, 15.02, 17.77, 20.81, 22.0, 22.99]
a, b = linreg(x, y)          # Linear regression
plot(x, y, "o")              # Plot (x, y) points
plot(x, a + b*x)             # Plot line determined by linear regression
\end{verbatim}

See also:

\texttt{\textbackslash{}}, \texttt{cov}, \texttt{std}, \texttt{mean}

    

    \begin{Verbatim}[commandchars=\\\{\}]
{\color{incolor}In [{\color{incolor}44}]:} \PY{c}{\PYZsh{} y = a + b * x}
         \PY{n}{x} \PY{o}{=} \PY{l+m+mf}{1.0}\PY{p}{:}\PY{l+m+mf}{12.0}
         \PY{n}{y} \PY{o}{=} \PY{p}{[}\PY{l+m+mf}{5.5}\PY{p}{,} \PY{l+m+mf}{6.3}\PY{p}{,} \PY{l+m+mf}{7.6}\PY{p}{,} \PY{l+m+mf}{8.8}\PY{p}{,} \PY{l+m+mf}{10.9}\PY{p}{,} \PY{l+m+mf}{11.79}\PY{p}{,} \PY{l+m+mf}{13.48}\PY{p}{,} \PY{l+m+mf}{15.02}\PY{p}{,} \PY{l+m+mf}{17.77}\PY{p}{,} \PY{l+m+mf}{20.81}\PY{p}{,} \PY{l+m+mf}{22.0}\PY{p}{,} \PY{l+m+mf}{22.99}\PY{p}{]}
         \PY{n}{a}\PY{p}{,} \PY{n}{b} \PY{o}{=} \PY{n}{linreg}\PY{p}{(}\PY{n}{x}\PY{p}{,} \PY{n}{y}\PY{p}{)}
\end{Verbatim}

            \begin{Verbatim}[commandchars=\\\{\}]
{\color{outcolor}Out[{\color{outcolor}44}]:} (2.5559090909090916,1.696013986013986)
\end{Verbatim}
        
    求めた結果を使ってプロットしてみます。

    \begin{Verbatim}[commandchars=\\\{\}]
{\color{incolor}In [{\color{incolor}45}]:} \PY{n}{Plots}\PY{o}{.}\PY{n}{plot}\PY{p}{(}\PY{n}{x}\PY{p}{,}\PY{n}{y}\PY{p}{,} \PY{n}{linetype}\PY{o}{=}\PY{p}{:}\PY{n}{scatter}\PY{p}{)}
         \PY{n}{Plots}\PY{o}{.}\PY{n}{plot!}\PY{p}{(}\PY{n}{x}\PY{p}{,} \PY{n}{a} \PY{o}{+} \PY{n}{b}\PY{o}{*}\PY{n}{x}\PY{p}{)}
\end{Verbatim}

    \begin{Verbatim}[commandchars=\\\{\}]
{\color{incolor}In [{\color{incolor} }]:} 
\end{Verbatim}

    \protect\hyperlink{ux76eeux6b21}{目次に戻る}

    \section{アンスコムの例}\label{ux30a2ux30f3ux30b9ux30b3ux30e0ux306eux4f8b}

データ分析において可視化することがいかに重要かということを知ることの出来る良い例と\href{https://en.wikipedia.org/wiki/Anscombe\%27s_quartet}{アンスコムの例(Anscombe's
quartet)}というものがあります。

アンスコムの例は4つのデータセットからなり、それぞれのデータセットの平均や分散、回帰直線などはほとんど同じなのに、散布図にすると似ても似つかない分布になるという面白い例です。

アンスコムの例のデータセットは統計ソフト R
のデータセットから読み込むことが出来ます。

    \begin{Verbatim}[commandchars=\\\{\}]
{\color{incolor}In [{\color{incolor}46}]:} \PY{k}{using} \PY{n}{RDatasets}
         \PY{n}{anscombe} \PY{o}{=} \PY{n}{RDatasets}\PY{o}{.}\PY{n}{dataset}\PY{p}{(}\PY{l+s}{\PYZdq{}}\PY{l+s}{datasets}\PY{l+s}{\PYZdq{}}\PY{p}{,}\PY{l+s}{\PYZdq{}}\PY{l+s}{anscombe}\PY{l+s}{\PYZdq{}}\PY{p}{)}
\end{Verbatim}

            \begin{Verbatim}[commandchars=\\\{\}]
{\color{outcolor}Out[{\color{outcolor}46}]:} 11×8 DataFrames.DataFrame
         │ Row │ X1 │ X2 │ X3 │ X4 │ Y1    │ Y2   │ Y3    │ Y4   │
         ├─────┼────┼────┼────┼────┼───────┼──────┼───────┼──────┤
         │ 1   │ 10 │ 10 │ 10 │ 8  │ 8.04  │ 9.14 │ 7.46  │ 6.58 │
         │ 2   │ 8  │ 8  │ 8  │ 8  │ 6.95  │ 8.14 │ 6.77  │ 5.76 │
         │ 3   │ 13 │ 13 │ 13 │ 8  │ 7.58  │ 8.74 │ 12.74 │ 7.71 │
         │ 4   │ 9  │ 9  │ 9  │ 8  │ 8.81  │ 8.77 │ 7.11  │ 8.84 │
         │ 5   │ 11 │ 11 │ 11 │ 8  │ 8.33  │ 9.26 │ 7.81  │ 8.47 │
         │ 6   │ 14 │ 14 │ 14 │ 8  │ 9.96  │ 8.1  │ 8.84  │ 7.04 │
         │ 7   │ 6  │ 6  │ 6  │ 8  │ 7.24  │ 6.13 │ 6.08  │ 5.25 │
         │ 8   │ 4  │ 4  │ 4  │ 19 │ 4.26  │ 3.1  │ 5.39  │ 12.5 │
         │ 9   │ 12 │ 12 │ 12 │ 8  │ 10.84 │ 9.13 │ 8.15  │ 5.56 │
         │ 10  │ 7  │ 7  │ 7  │ 8  │ 4.82  │ 7.26 │ 6.42  │ 7.91 │
         │ 11  │ 5  │ 5  │ 5  │ 8  │ 5.68  │ 4.74 │ 5.73  │ 6.89 │
\end{Verbatim}
        
    ここで、読み込んだデータは DataFrame
という配列に似たものです。詳細は次回。
まずは、4つのデータをプロットしてみます。

DaatFrame 型は各列を

\begin{Shaded}
\begin{Highlighting}[]
\NormalTok{anscombe[:X1]}
\end{Highlighting}
\end{Shaded}

のように列名で抜き出すことが出来ます。

    \begin{Verbatim}[commandchars=\\\{\}]
{\color{incolor}In [{\color{incolor}47}]:} \PY{n}{Plots}\PY{o}{.}\PY{n}{scatter}\PY{p}{(}\PY{n}{anscombe}\PY{p}{[}\PY{p}{:}\PY{n}{X1}\PY{p}{]}\PY{p}{,} \PY{n}{anscombe}\PY{p}{[}\PY{p}{:}\PY{n}{Y1}\PY{p}{]}\PY{p}{,} \PY{n}{xlims}\PY{o}{=}\PY{p}{(}\PY{l+m+mi}{0}\PY{p}{,}\PY{l+m+mi}{20}\PY{p}{)}\PY{p}{)}
\end{Verbatim}

    \begin{Verbatim}[commandchars=\\\{\}]
{\color{incolor}In [{\color{incolor}48}]:} \PY{n}{Plots}\PY{o}{.}\PY{n}{scatter}\PY{p}{(}\PY{n}{anscombe}\PY{p}{[}\PY{p}{:}\PY{n}{X2}\PY{p}{]}\PY{p}{,} \PY{n}{anscombe}\PY{p}{[}\PY{p}{:}\PY{n}{Y2}\PY{p}{]}\PY{p}{,} \PY{n}{xlims}\PY{o}{=}\PY{p}{(}\PY{l+m+mi}{0}\PY{p}{,}\PY{l+m+mi}{20}\PY{p}{)}\PY{p}{)}
\end{Verbatim}

    \begin{Verbatim}[commandchars=\\\{\}]
{\color{incolor}In [{\color{incolor}49}]:} \PY{n}{Plots}\PY{o}{.}\PY{n}{scatter}\PY{p}{(}\PY{n}{anscombe}\PY{p}{[}\PY{p}{:}\PY{n}{X3}\PY{p}{]}\PY{p}{,} \PY{n}{anscombe}\PY{p}{[}\PY{p}{:}\PY{n}{Y3}\PY{p}{]}\PY{p}{,} \PY{n}{xlims}\PY{o}{=}\PY{p}{(}\PY{l+m+mi}{0}\PY{p}{,}\PY{l+m+mi}{20}\PY{p}{)}\PY{p}{)}
\end{Verbatim}

    \begin{Verbatim}[commandchars=\\\{\}]
{\color{incolor}In [{\color{incolor}50}]:} \PY{n}{Plots}\PY{o}{.}\PY{n}{scatter}\PY{p}{(}\PY{n}{anscombe}\PY{p}{[}\PY{p}{:}\PY{n}{X4}\PY{p}{]}\PY{p}{,} \PY{n}{anscombe}\PY{p}{[}\PY{p}{:}\PY{n}{Y4}\PY{p}{]}\PY{p}{,} \PY{n}{xlims}\PY{o}{=}\PY{p}{(}\PY{l+m+mi}{0}\PY{p}{,}\PY{l+m+mi}{20}\PY{p}{)}\PY{p}{)}
\end{Verbatim}

    4つの散布図は似ても似つきません。しかし、統計量は非常に近い値を取ります。

    \begin{Verbatim}[commandchars=\\\{\}]
{\color{incolor}In [{\color{incolor}51}]:} \PY{c}{\PYZsh{} Introducing Julia/DataFrames \PYZhy{} Wikibooks, open books for an open world }
         \PY{c}{\PYZsh{} https://en.wikibooks.org/wiki/Introducing\PYZus{}Julia/DataFrames\PYZsh{}Plotting\PYZus{}Anscombe.27s\PYZus{}Quartet}
         
         \PY{c}{\PYZsh{} print a header}
         \PY{n}{println}\PY{p}{(}\PY{l+s}{\PYZdq{}}\PY{l+s}{Column}\PY{l+s+se}{\PYZbs{}t}\PY{l+s}{MeanX}\PY{l+s+se}{\PYZbs{}t}\PY{l+s}{MedianX}\PY{l+s+se}{\PYZbs{}t}\PY{l+s}{StdDev X}\PY{l+s+se}{\PYZbs{}t}\PY{l+s}{MeanY}\PY{l+s+se}{\PYZbs{}t}\PY{l+s+se}{\PYZbs{}t}\PY{l+s+se}{\PYZbs{}t}\PY{l+s}{StdDev Y}\PY{l+s+se}{\PYZbs{}t}\PY{l+s+se}{\PYZbs{}t}\PY{l+s}{Corr}\PY{l+s+se}{\PYZbs{}t}\PY{l+s}{\PYZdq{}}\PY{p}{)}
         \PY{n}{map}\PY{p}{(}\PY{p}{(}\PY{n}{xcol}\PY{p}{,}\PY{n}{ycol}\PY{p}{)} \PY{o}{\PYZhy{}\PYZgt{}} \PY{n}{println}\PY{p}{(}
             \PY{n}{xcol}\PY{p}{,}                   \PY{l+s}{\PYZdq{}}\PY{l+s+se}{\PYZbs{}t}\PY{l+s}{\PYZdq{}}\PY{p}{,}
             \PY{n}{mean}\PY{p}{(}\PY{n}{anscombe}\PY{p}{[}\PY{n}{xcol}\PY{p}{]}\PY{p}{)}\PY{p}{,}   \PY{l+s}{\PYZdq{}}\PY{l+s+se}{\PYZbs{}t}\PY{l+s}{\PYZdq{}}\PY{p}{,} 
             \PY{n}{median}\PY{p}{(}\PY{n}{anscombe}\PY{p}{[}\PY{n}{xcol}\PY{p}{]}\PY{p}{)}\PY{p}{,} \PY{l+s}{\PYZdq{}}\PY{l+s+se}{\PYZbs{}t}\PY{l+s}{\PYZdq{}}\PY{p}{,} 
             \PY{n}{std}\PY{p}{(}\PY{n}{anscombe}\PY{p}{[}\PY{n}{xcol}\PY{p}{]}\PY{p}{)}\PY{p}{,}    \PY{l+s}{\PYZdq{}}\PY{l+s+se}{\PYZbs{}t}\PY{l+s}{\PYZdq{}}\PY{p}{,} 
             \PY{n}{mean}\PY{p}{(}\PY{n}{anscombe}\PY{p}{[}\PY{n}{ycol}\PY{p}{]}\PY{p}{)}\PY{p}{,}   \PY{l+s}{\PYZdq{}}\PY{l+s+se}{\PYZbs{}t}\PY{l+s}{\PYZdq{}}\PY{p}{,} 
             \PY{n}{std}\PY{p}{(}\PY{n}{anscombe}\PY{p}{[}\PY{n}{ycol}\PY{p}{]}\PY{p}{)}\PY{p}{,}    \PY{l+s}{\PYZdq{}}\PY{l+s+se}{\PYZbs{}t}\PY{l+s}{\PYZdq{}}\PY{p}{,} 
             \PY{n}{cor}\PY{p}{(}\PY{n}{anscombe}\PY{p}{[}\PY{n}{xcol}\PY{p}{]}\PY{p}{,} \PY{n}{anscombe}\PY{p}{[}\PY{n}{ycol}\PY{p}{]}\PY{p}{)}\PY{p}{)}\PY{p}{,} 
             
             \PY{p}{[}\PY{p}{:}\PY{n}{X1}\PY{p}{,} \PY{p}{:}\PY{n}{X2}\PY{p}{,} \PY{p}{:}\PY{n}{X3}\PY{p}{,} \PY{p}{:}\PY{n}{X4}\PY{p}{]}\PY{p}{,} 
             \PY{p}{[}\PY{p}{:}\PY{n}{Y1}\PY{p}{,} \PY{p}{:}\PY{n}{Y2}\PY{p}{,} \PY{p}{:}\PY{n}{Y3}\PY{p}{,} \PY{p}{:}\PY{n}{Y4}\PY{p}{]}\PY{p}{)}\PY{p}{;}
\end{Verbatim}

    \begin{Verbatim}[commandchars=\\\{\}]
Column	MeanX	MedianX	StdDev X	MeanY			StdDev Y		Corr	
X1	9.0	9.0	3.3166247903554	7.500909090909093	2.031568135925815	0.8164205163448398
X2	9.0	9.0	3.3166247903554	7.500909090909091	2.0316567355016177	0.8162365060002427
X3	9.0	9.0	3.3166247903554	7.500000000000001	2.030423601123667	0.8162867394895983
X4	9.0	8.0	3.3166247903554	7.50090909090909	2.0305785113876023	0.8165214368885029

    \end{Verbatim}

    \begin{Verbatim}[commandchars=\\\{\}]
{\color{incolor}In [{\color{incolor} }]:} 
\end{Verbatim}

    \protect\hyperlink{ux76eeux6b21}{目次に戻る}

    \section{練習問題}\label{ux7df4ux7fd2ux554fux984c}

    \subsection{1.}\label{section}

線形方程式 \$

\begin{align}
  \left\{
    \begin{array}{l}
      x + 2y = -1 \\
      3x + y = 2
    \end{array}
  \right.
\end{align}

\$ を解け

    \begin{Verbatim}[commandchars=\\\{\}]
{\color{incolor}In [{\color{incolor} }]:} 
\end{Verbatim}

    \subsection{2.}\label{section}

以下のコード

\begin{Shaded}
\begin{Highlighting}[]
    \NormalTok{srand(}\FloatTok{1}\NormalTok{)}
    \NormalTok{scores = rand(}\FloatTok{0}\NormalTok{:}\FloatTok{100}\NormalTok{, }\FloatTok{100}\NormalTok{, }\FloatTok{3}\NormalTok{)}
\end{Highlighting}
\end{Shaded}

を実行し、配列 scores の 1. 各列の合計を求めよ。 1. 各列の平均を求めよ。
1. 各列の分散を求めよ。

    \begin{Verbatim}[commandchars=\\\{\}]
{\color{incolor}In [{\color{incolor} }]:} 
\end{Verbatim}

    \subsection{3.}\label{section}

次のコードを実行するとタイタニックの乗客乗員の情報を読み込むことが出来る。

\begin{Shaded}
\begin{Highlighting}[]
\NormalTok{using RDatasets}
\NormalTok{titanic = RDatasets.dataset(}\StringTok{"COUNT"}\NormalTok{, }\StringTok{"titanic"}\NormalTok{)}
\end{Highlighting}
\end{Shaded}

\begin{longtable}[c]{@{}lllll@{}}
\toprule
& Survived & Age & Sex & Class\tabularnewline
\midrule
\endhead
1 & 1 & 1 & 1 & 1\tabularnewline
2 & 2 & 1 & 1 & 1\tabularnewline
3 & 3 & 1 & 1 & 1\tabularnewline
4 & 4 & 1 & 1 & 1\tabularnewline
5 & 5 & 1 & 1 & 1\tabularnewline
6 & 6 & 1 & 1 & 1\tabularnewline
\bottomrule
\end{longtable}

ここで各列の数字の意味は以下のとおりである。 Survived 1=survived; 0=died

Age 1=adult; 0=child

Sex 1=Male; 0=female

Class ticket class 1= 1st class; 2= second class; 3= third class

\begin{enumerate}
\def\labelenumi{\arabic{enumi}.}
\tightlist
\item
  生存者と死者の数をそれぞれ調べよ。
\item
  男女別に生存者と死者の数をそれぞれ調べよ。
\item
  年齢別に生存者と死者の数をそれぞれ調べよ。
\item
  チケットの階級別に生存者と死者の数をそれぞれ調べよ。
\item
  1〜4 の結果を図示せよ。
\end{enumerate}

    \begin{Verbatim}[commandchars=\\\{\}]
{\color{incolor}In [{\color{incolor}52}]:} \PY{k}{using} \PY{n}{RDatasets}
         \PY{n}{titanic} \PY{o}{=} \PY{n}{RDatasets}\PY{o}{.}\PY{n}{dataset}\PY{p}{(}\PY{l+s}{\PYZdq{}}\PY{l+s}{COUNT}\PY{l+s}{\PYZdq{}}\PY{p}{,} \PY{l+s}{\PYZdq{}}\PY{l+s}{titanic}\PY{l+s}{\PYZdq{}}\PY{p}{)}
         \PY{n}{head}\PY{p}{(}\PY{n}{titanic}\PY{p}{)}
\end{Verbatim}

            \begin{Verbatim}[commandchars=\\\{\}]
{\color{outcolor}Out[{\color{outcolor}52}]:} 6×4 DataFrames.DataFrame
         │ Row │ Survived │ Age │ Sex │ Class │
         ├─────┼──────────┼─────┼─────┼───────┤
         │ 1   │ 1        │ 1   │ 1   │ 1     │
         │ 2   │ 1        │ 1   │ 1   │ 1     │
         │ 3   │ 1        │ 1   │ 1   │ 1     │
         │ 4   │ 1        │ 1   │ 1   │ 1     │
         │ 5   │ 1        │ 1   │ 1   │ 1     │
         │ 6   │ 1        │ 1   │ 1   │ 1     │
\end{Verbatim}
        
    \begin{Verbatim}[commandchars=\\\{\}]
{\color{incolor}In [{\color{incolor} }]:} 
\end{Verbatim}

    \subsection{4.}\label{section}

福岡市の月ごとの降水量を調べ棒グラフで図示せよ。

    \begin{Verbatim}[commandchars=\\\{\}]
{\color{incolor}In [{\color{incolor} }]:} 
\end{Verbatim}

    \protect\hyperlink{ux76eeux6b21}{目次に戻る}


    % Add a bibliography block to the postdoc
    
    
    
    \end{document}
